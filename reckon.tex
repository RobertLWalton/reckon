% Personal Calculation and Simulation Langauge (PCASL)
%
% File:         pcasl.tex
% Author:       Bob Walton (walton@deas.harvard.edu)
% Date:		See \date below.
  
\documentclass[12pt]{article}

\usepackage{makeidx}

\makeindex

\setlength{\oddsidemargin}{0in}
\setlength{\evensidemargin}{0in}
\setlength{\textwidth}{6.5in}
\raggedbottom

\pagestyle{headings}
\setlength{\parindent}{0.0in}
\setlength{\parskip}{1ex}

\setcounter{secnumdepth}{5}
\setcounter{tocdepth}{5}
\newcommand{\subsubsubsection}[1]{\paragraph[#1]{#1.}}
\newcommand{\subsubsubsubsection}[1]{\subparagraph[#1]{#1.}}

% Begin \tableofcontents surgery.

\newcount\ATCATCODE
\ATCATCODE=\catcode`@
\catcode `@=11	% @ is now a letter

\renewcommand{\contentsname}{}
\renewcommand\l@section{\@dottedtocline{1}{0.1em}{1.4em}}
\renewcommand\l@table{\@dottedtocline{1}{0.1em}{1.4em}}
\renewcommand\tableofcontents{%
    \begin{list}{}%
	     {\setlength{\itemsep}{0in}%
	      \setlength{\topsep}{0in}%
	      \setlength{\parsep}{1ex}%
	      \setlength{\labelwidth}{0in}%
	      \setlength{\baselineskip}{1.5ex}%
	      \setlength{\leftmargin}{1.0in}%
	      \setlength{\rightmargin}{1.0in}}%
    \item\@starttoc{toc}%
    \end{list}}
\renewcommand\listoftables{%
    \begin{list}{}%
	     {\setlength{\itemsep}{0in}%
	      \setlength{\topsep}{0in}%
	      \setlength{\parsep}{1ex}%
	      \setlength{\labelwidth}{0in}%
	      \setlength{\baselineskip}{1.5ex}%
	      \setlength{\leftmargin}{1.0in}%
	      \setlength{\rightmargin}{1.0in}%
	      }%
    \item\@starttoc{lot}%
    \end{list}}

\catcode `@=\ATCATCODE	% @ is now restored

% End \tableofcontents surgery.

\newcommand{\CN}[2]%	Change Notice.
    {\hspace*{0in}\marginpar{\sloppy \raggedright \it \footnotesize
     $^{\mbox{#1}}$#2}}
    % Change notice.

\newcommand{\key}[1]{{\em #1}\index{#1}}
\newcommand{\mkey}[2]{{\em #1}\index{#1!#2}}
\newcommand{\skey}[2]{{\em #1#2}\index{#1}}
\newcommand{\ikey}[2]{{\em #1}\index{#2}}
\newcommand{\ttkey}[1]{{\tt #1}\index{#1@{\tt #1}}}
\newcommand{\ttmkey}[2]{{\tt #1}\index{#1@{\tt #1}!#2}}
\newcommand{\ttfkey}[2]{{\tt #1}\index{#1@{\tt #1}!for #2@for {\tt #2}}}
\newcommand{\ttakey}[2]{{\tt #1}\index{#2@{\tt #1}}}
\newcommand{\ttamkey}[3]{{\tt #1}\index{#2@{\tt #1}!#3}}
\newcommand{\ttindex}[1]{\index{#1@{\tt #1}}}
\newcommand{\ttmindex}[2]{\index{#1@{\tt #1}!#2}}
\newcommand{\emkey}[1]{{\em #1}\index{#1@{\em #1}}}
\newcommand{\emindex}[1]{\index{#1@{\em #1}}}

\newcommand{\EOL}{\penalty \exhyphenpenalty}

\newsavebox{\leftbracket}
\begin{lrbox}{\leftbracket}
\verb|{|
\end{lrbox}

\newsavebox{\rightbracket}
\begin{lrbox}{\rightbracket}
\verb|}|
\end{lrbox}

\newcommand{\ttbrackets}{
    \renewcommand{\{}{\usebox{\leftbracket}}
    \renewcommand{\}}{\usebox{\rightbracket}}}

\newlength{\figurewidth}
\setlength{\figurewidth}{\textwidth}
\addtolength{\figurewidth}{-0.40in}

\newsavebox{\figurebox}

\newenvironment{boxedfigure}[1][!btp]%
	{\begin{figure*}[#1]
	 \begin{lrbox}{\figurebox}
	 \begin{minipage}{\figurewidth}

	 \vspace*{1ex}}%
	{
	 \vspace*{1ex}

	 \end{minipage}
	 \end{lrbox}
	 \begin{center}
	 \fbox{\hspace*{0.1in}\usebox{\figurebox}\hspace*{0.1in}}
	 \end{center}
	 \end{figure*}}

\newenvironment{indpar}[1][0.3in]%
	{\begin{list}{}%
		     {\setlength{\itemsep}{0in}%
		      \setlength{\topsep}{0in}%
		      \setlength{\parsep}{1ex}%
		      \setlength{\labelwidth}{#1}%
		      \setlength{\leftmargin}{#1}%
		      \addtolength{\leftmargin}{\labelsep}}%
	 \item}%
	{\end{list}}

\begin{document}
        
\title{Personal\\Calculation and Simulation\\Language\\[2ex]PCASL\\[2ex]
       (Draft 1a)}

\author{Robert L. Walton\thanks{This document is was partly inspired
teaching courses at Suffolk University.}}

\date{July 10, 2003}
 
\maketitle

\tableofcontents 

\newpage

\section{Introduction}

This document describes PCASL, the Personal Calculation and Simulation
Language, that is informally referred to as P-Castle, Personal Castle, or just
Castle.

PCASL is designed for naive programmers: that is, for people who may never
be able to program computers well.  It is a simple language that has
powerful data types which make it easier to write small programs
that do a variety of tasks that a person might want to do.  Generally
the tasks fall into the categories of calculating things (taxes,
statistics) or simulating things (computer games).  Included are:

\begin{center}
\begin{tabular}{l}
Calculations that might be done with a spreadsheet. \\
Drawing pictures. \\
Simulating popular board games and creating new ones. \\
Creating simple computer games, including dialog games. \\
Analysing documents. \\
Doing elementary algebra and calculus problems. \\
Calculating basic statistics. \\
Simulating simple electrical, mechanical, and chemical systems.\\
Solving problems in elementary logic. \\
\end{tabular}
\end{center}

There are many computer languages that have some powerful data type that adapts
them for a specific kind of computation.  PCASL tries to combine these.
Some previous computer languages that have influenced PCASL are:

\begin{center}
\begin{tabular}{l@{\hspace{0.5in}}l}
Various Spreadsheets	& Spreadsheets \\
Matlab			& Matrices. \\
Mathematica		& Expressions \\
LISP			& Words and Phrases \\
TCL			& Character Strings and Lists \\
\end{tabular}
\end{center}

PCASL is \underline{not} designed to be a computer-efficient language.
It is designed to be person-efficient, and to do small calculations
rapidly enough with inexpensive modern computers.

\section{Data}

PCASL has two major kinds of data: expressions and blocks.  Numbers are
the simplest expressions.  More complex expressions are math
expressions or document expressions.  Blocks are sets of variables
each of which can have a value, which is an expression, and also
a definition, which is another expression that is used to compute the
value when the value is needed.  The definitions of a block, taken all
together, are called the `code' of the block.

You can use PCASL as a calculator by typing into it expressions to
be evaluated and assignments of values and definitions to variables.
Some examples involving numbers are:

\begin{indpar}\begin{verbatim}
> 9
9
> 9 + 8
17
> x = 9
9
> y = 9 + 8
17
> x + y
26
\end{verbatim}\end{indpar}

Here the `\verb|> |' at the beginning of some lines is the PCASL \key{prompt}
that tells you its OK to input an expression to be evaluated.

At somewhat the opposite extreme from numbers are words, phrases, sentences,
and paragraphs.  You can calculate with these `\skey{document expression}s' too:

\begin{indpar}
\verb|> g = `hello'| \\
\verb|`|hello\verb|'| \\
\verb|> `<g> there'| \\
\verb|`|hello there\verb|'| \\
\verb|> z = ``I thought he said `<g>'.''| \\
\verb|``|I thought he said `hello'.\verb|''| \\
\verb|> notice = ``This document is meant to be read.| \\
\verb|             Reading this document is good, but...| \\
\verb|             <z>.''| \\
\verb|``|This document is meant to be read. \\
\verb|  |Reading this document is good, but\ldots \\
\verb|  |I thought he said `hello'.\verb|''| \\
\verb|> `When you add <x> and <y> you get <x+y>.'| \\
\verb|`|When you add 9 and 17 you get 26.\verb|'|
\end{indpar}

Modern math computes with expressions, and not just numbers.
You can compute with \skey{math expression}s in PCASL too:

\begin{indpar}
\verb|> f = [10x^2 - 3.67x - 0.04]| \\
\verb|[|$10x^2-3.67x-0.04$\verb|]| \\
\verb|> h = [- 0.96 + 0.67x]| \\
\verb|[|$-0.96+0.67x$\verb|]| \\
\verb|> f + h| \\
\verb|[|$10x^2-3x-1$\verb|]| \\
\verb|> solve [f + h = 0]| \\
\verb|[x = (-0.2, 0.5)]| \\
\verb|> evaluate [f + h] at [x = (3, 4, 5)]| \\
\verb|(78.95, 145.28, 231.61)| \\
\verb|> g = [integral (x^2 dx)]| \\
\verb|[|$\int x^2 dx$\verb|]| \\
\verb|> simplify g| \\
\verb|[|$\frac{1}{3} x^3$\verb|]| \\
\verb|> v = evaluate g from [x=1] to [x=5]| \\
\verb|41 1/3| \\
\verb|> `The value of <definite g from [x=1] to [x=5]> is <v>.'| \\
\verb|`|The value of $\int_{x=1}^{x=5} x^2 dx$ is $41\frac{1}{3}$.\verb|'|
\end{indpar}



\bibliographystyle{plain}
\bibliography{pcasl}

\printindex

\end{document}

