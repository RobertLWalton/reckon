% Personal Calculation and Simulation Langauge (PCASL)
%
% File:         pcasl.tex
% Author:       Bob Walton (walton@deas.harvard.edu)
% Date:		See \date below.
  
\documentclass[12pt]{article}

\usepackage{makeidx}
\usepackage{pictex}

\makeindex

\setlength{\oddsidemargin}{0in}
\setlength{\evensidemargin}{0in}
\setlength{\textwidth}{6.5in}
\raggedbottom

\setlength{\unitlength}{1in}

\pagestyle{headings}
\setlength{\parindent}{0.0in}
\setlength{\parskip}{1ex}

\setcounter{secnumdepth}{5}
\setcounter{tocdepth}{5}
\newcommand{\subsubsubsection}[1]{\paragraph[#1]{#1.}}
\newcommand{\subsubsubsubsection}[1]{\subparagraph[#1]{#1.}}

% Begin \tableofcontents surgery.

\newcount\AtCatcode
\AtCatcode=\catcode`@
\catcode `@=11	% @ is now a letter

\renewcommand{\contentsname}{}
\renewcommand\l@section{\@dottedtocline{1}{0.1em}{1.4em}}
\renewcommand\l@table{\@dottedtocline{1}{0.1em}{1.4em}}
\renewcommand\tableofcontents{%
    \begin{list}{}%
	     {\setlength{\itemsep}{0in}%
	      \setlength{\topsep}{0in}%
	      \setlength{\parsep}{1ex}%
	      \setlength{\labelwidth}{0in}%
	      \setlength{\baselineskip}{1.5ex}%
	      \setlength{\leftmargin}{1.0in}%
	      \setlength{\rightmargin}{1.0in}}%
    \item\@starttoc{toc}%
    \end{list}}
\renewcommand\listoftables{%
    \begin{list}{}%
	     {\setlength{\itemsep}{0in}%
	      \setlength{\topsep}{0in}%
	      \setlength{\parsep}{1ex}%
	      \setlength{\labelwidth}{0in}%
	      \setlength{\baselineskip}{1.5ex}%
	      \setlength{\leftmargin}{1.0in}%
	      \setlength{\rightmargin}{1.0in}%
	      }%
    \item\@starttoc{lot}%
    \end{list}}

\catcode `@=\AtCatcode	% @ is now restored

% End \tableofcontents surgery.

\newcommand{\CN}[2]%	Change Notice.
    {\hspace*{0in}\marginpar{\sloppy \raggedright \it \footnotesize
     $^{\mbox{#1}}$#2}}
    % Change notice.

\newcommand{\key}[1]{{\em #1}\index{#1}}
\newcommand{\mkey}[2]{{\em #1}\index{#1!#2}}
\newcommand{\skey}[2]{{\em #1#2}\index{#1}}
\newcommand{\ikey}[2]{{\em #1}\index{#2}}
\newcommand{\ttkey}[1]{{\tt #1}\index{#1@{\tt #1}}}
\newcommand{\ttmkey}[2]{{\tt #1}\index{#1@{\tt #1}!#2}}
\newcommand{\ttfkey}[2]{{\tt #1}\index{#1@{\tt #1}!for #2@for {\tt #2}}}
\newcommand{\ttakey}[2]{{\tt #1}\index{#2@{\tt #1}}}
\newcommand{\ttamkey}[3]{{\tt #1}\index{#2@{\tt #1}!#3}}
\newcommand{\ttindex}[1]{\index{#1@{\tt #1}}}
\newcommand{\ttmindex}[2]{\index{#1@{\tt #1}!#2}}
\newcommand{\emkey}[1]{{\em #1}\index{#1@{\em #1}}}
\newcommand{\emindex}[1]{\index{#1@{\em #1}}}

\newcommand{\secref}[1]{\ref{#1}{ p\pageref{#1}}}
\newcommand{\Secref}[1]{\ref{#1}{(p\pageref{#1})}}
\newcommand{\stepref}[1]{\ref{#1}{(p\pageref{#1})}}
\newcommand{\appref}[1]{\ref{#1}{ p\pageref{#1}}}
\newcommand{\pagref}[1]{p\pageref{#1}}

\newcommand{\EOL}{\penalty \exhyphenpenalty}

\newcount\TildeCatcode
\TildeCatcode=\catcode`\~
\catcode`~=12
\newcommand{\Tilde}{~}
\catcode`~=\TildeCatcode

\newcount\CircumflexCatcode
\CircumflexCatcode=\catcode`\^
\catcode`^=12
\newcommand{\Circumflex}{^}
\catcode`^=\CircumflexCatcode

\newcount\CurlyBraCatcode
\newcount\CurlyKetCatcode
\newcount\SquareBraCatcode
\newcount\SquareKetCatcode
\CurlyBraCatcode=\catcode`{
\CurlyKetCatcode=\catcode`}
\SquareBraCatcode=\catcode`[
\SquareKetCatcode=\catcode`]

\catcode`{=\SquareBraCatcode
\catcode`}=\SquareKetCatcode
\catcode`[=\CurlyBraCatcode
\catcode`]=\CurlyKetCatcode

\newcommand[\CurlyBra][{]
\newcommand[\CurlyKet][}]

\catcode`{=\CurlyBraCatcode
\catcode`}=\CurlyKetCatcode
\catcode`[=\SquareBraCatcode
\catcode`]=\SquareKetCatcode

\newcommand{\ttbrackets}{%
    \renewcommand{\{}{\CurlyBra}%
    \renewcommand{\}}{\CurlyKet}}

\newlength{\figurewidth}
\setlength{\figurewidth}{\textwidth}
\addtolength{\figurewidth}{-0.40in}

\newsavebox{\figurebox}

\newenvironment{boxedfigure}[1][!btp]%
	{\begin{figure*}[#1]
	 \begin{lrbox}{\figurebox}
	 \begin{minipage}{\figurewidth}

	 \vspace*{1ex}}%
	{
	 \vspace*{1ex}

	 \end{minipage}
	 \end{lrbox}
	 \begin{center}
	 \fbox{\hspace*{0.1in}\usebox{\figurebox}\hspace*{0.1in}}
	 \end{center}
	 \end{figure*}}

\newenvironment{indpar}[1][0.3in]%
	{\begin{list}{}%
		     {\setlength{\itemsep}{0in}%
		      \setlength{\topsep}{0in}%
		      \setlength{\parsep}{1ex}%
		      \setlength{\labelwidth}{#1}%
		      \setlength{\leftmargin}{#1}%
		      \addtolength{\leftmargin}{\labelsep}}%
	 \item}%
	{\end{list}}

\begin{document}
        
\title{Personal\\Calculation and Simulation\\Language\\[2ex]PCASL\\[2ex]
       (Draft 1a)}

\author{Robert L. Walton\thanks{Copyright 2004 Robert L. Walton.
Permission to copy this document verbatim is granted by the author
to the public.  This document was partly inspired
by teaching courses at Suffolk University, and by the work of Thomas
Cheatham and Stuart Shieber.}}

\date{May 20, 2004}
 
\maketitle

\newpage
\begin{center}
\large \bf Table of Contents
\end{center}

\bigskip

\tableofcontents 

\newpage

\section{Introduction}

This document describes PCASL, the Personal Calculation and Simulation
Language, that is informally referred to as P-Castle, Personal Castle, or just
Castle.

PCASL is designed for naive programmers: that is, for people who may never
be able to program computers well.  It is a simple language that has
powerful data types which make it easier to write small programs
that do a variety of tasks that a person might want to do.  Generally
the tasks fall into the categories of calculating things (taxes,
probabilities, statistics) or simulating things
(computer games, construction designs, mathematics demonstrations).
Included are:

\begin{center}
\begin{tabular}{l}
Calculations that might be done with a spreadsheet. \\
Drawing pictures. \\
Simulating popular board games and creating new ones. \\
Creating simple computer games, including dialog games. \\
Computing and analyzing documents. \\
Doing elementary algebra and calculus problems. \\
Calculating basic probabilities and statistics. \\
Simulating two and three dimensional objects. \\
Simulating simple electrical, mechanical, chemical, and biological systems.\\
Solving problems in elementary logic. \\
\end{tabular}
\end{center}

There are many computer languages that have some powerful data type that adapts
them for a specific kind of computation.  PCASL tries to combine these.
Some previous computer languages that have influenced PCASL, and
the data types they particularly support, are:

\begin{center}
\begin{tabular}{l@{\hspace{0.5in}}l}
Various Spreadsheets		& Spreadsheets \\
Various Data Base Languages	& Data Bases \\
Various Script Languages	& Documents \\
MATLAB				& Matrices \\
Mathematica			& Expressions \\
TCL				& Character Strings and Lists \\
Lisp				& Words and Phrases \\
PROLOG				& Logical Expressions \\
\end{tabular}
\end{center}

PCASL is \underline{not} designed to be a computer-efficient language.
It is designed to be person-efficient, and to do small calculations
rapidly enough with inexpensive modern computers.

\section{Remarks}

PCASL was created as an answer to the question: what programming language
should you teach beginning programming students who do not have the talent
or inclination to become good programmers?  The initial answer, that it does
not matter provided you implement some powerful types of data in the
language you choose, has a flaw.  The flaw is that without the right powerful
data types, the language will be useless to the students after the course
is over.  So what is needed is a programming language that will be useful
to students after a first course in programming, and the essence of
such a language is the integration within it of many powerful and useful
data types.

The basic principles of the PCASL design were developed
by the author while teaching the intended customers of
PCASL.\footnote{Specifically, while teaching CS121 at Suffolk University.}
The language should have as few parts as possible, to cut down
on the amount of detail that must be remembered to use the language, but
conversely, there is no limit to the conceptual complexity of any well-used
part.\footnote{There was no problem teaching recursion, but it was better
not to each many different looping constructs.}
The language should have powerful data types, well integrated into
the syntax of the language.  As much as possible, statement executions
in the language should have visible effect.

The current version of PCASL is not stable, because it has not been
implemented, and because, unlike most programming languages, PCASL
has lots of subtle important interactions between its various features.  The
hope is that with implementation and experimentation, a stable sensible
version of PCASL integrating all its data types can be achieved.

\section{Overview}

PCASL has two major kinds of data: expressions and blocks.  Numbers are
the simplest expressions.  More complex expressions are math
expressions or document expressions.  Blocks are sets of variables
each of which can have a value, which is an expression, and also
a definition, which is a set of guarded expressions.
Each guarded expression consists of a guard, which is an expression that
evaluates to true or false, and a value expression, which is an expression
that is evaluated if the guard is true to produce the value of the variable.
The definitions of a block, taken all
together, are called the `code' of the block.

You can use PCASL as a calculator by typing into it expressions to
be evaluated and assignments of values and definitions to variables.
Some examples involving numbers are:

\begin{indpar}\begin{verbatim}
> 9
9
> 9 + 8
17
> x = 9
9
> y = 9 + 8
17
> x + y
26
\end{verbatim}\end{indpar}

Here the `\verb|> |' at the beginning of some lines is the PCASL \key{prompt}
that tells you its OK to input an expression to be evaluated.

At somewhat the opposite extreme from numbers are words, phrases, sentences,
and paragraphs.  You can calculate with these `\skey{document expression}s'.

\begin{indpar}
\verb|> g = `hello'| \\
\verb|`|hello\verb|'| \\
\verb|> `<<g>> there'| \\
\verb|`|hello there\verb|'| \\
\verb|> z = ``I thought he said `<<g>>'.''| \\
\verb|``|I thought he said `hello'.\verb|''| \\
\verb/> notice = ``|This document is meant to be read./ \\
\verb/+            |Reading this document is good, but.../ \\
\verb/+            |<<z>>.''/ \\
\verb|``|This document is meant to be read. \\
\verb|  |Reading this document is good, but\ldots \\
\verb|  |I thought he said `hello'.\verb|''| \\
\verb|> `When you add <<x>> and <<y>> you get <<x+y>>.'| \\
\verb|`|When you add 9 and 17 you get 26.\verb|'|
\end{indpar}

Modern math computes with expressions, and not just numbers.
You can compute with \skey{math expression}s in PCASL.

\begin{indpar}
\verb|> f = {10x^2 - 3.67x - 0.04}| \\
\verb|{|$10x^2-3.67x-0.04$\verb|}| \\
\verb|> h = (- 0.96 + 0.67x) in x| \\
\verb|{|$-0.96+0.67x$\verb|}| \\
\verb|> (f + h) in x| \\
\verb|{|$10x^2-3x-1$\verb|}| \\
\verb|> solve (f + h = 0) for x| \\
\verb|{|$x = (-0.2, 0.5)$\verb|}| \\
\verb|> (f + h) at (x = (3, 4, 5))| \\
\verb|(78.95, 145.28, 231.61)| \\
\verb|> g = {integral (x ^ 2 dx)}| \\
\verb|{|$\int x^2 dx$\verb|}| \\
\verb|> simplify g| \\
\verb|{|$\frac{1}{3} x^3$\verb|}| \\
\verb|> v = g from (x = 1) to (x = 5)| \\
\verb|41 1/3| \\
\verb|> out = `The value of {<<<g>>> from (x = 1) to (x = 5)} is <<v>>.'| \\
\verb|`|The value of $\int_{x = 1}^{x = 5} x^2 dx$ is $41\frac{1}{3}$.\verb|'| \\
\verb|> raw out| \\
\verb|[sentence the value of| \\
\verb|          {(integral (x ^ 2 * dx)) from (x = 1) to (x = 5)}| \\
\verb|          is 124/3]|
\end{indpar}

Another kind of datum you can compute with in PCASL is
the \key{block}:

\begin{indpar}
\verb|> a person {| \\
\verb|+     name = `Jack'| \\
\verb|+     weight = 123 lb| \\
\verb|+     height = 5' 9"| \\
\verb|+     age = 23 yr 2 mo }| \\
\begin{tabular}{|r|l|r|r|r|}
\hline
\multicolumn{1}{|c}{\bf ID} &
\multicolumn{1}{|c}{\bf name} &
\multicolumn{1}{|c}{\bf weight} &
\multicolumn{1}{|c}{\bf height} &
\multicolumn{1}{|c|}{\bf age} \\
\hline
\tt @1000000 & Jack & \tt 123 lb & \tt 5' 9" & \tt 23 yr 2 mo \\
\hline
\end{tabular} \\
\verb|> a person {| \\
\verb|+     name = `Jill'| \\
\verb|+     weight = 110 lb| \\
\verb|+     height = 5' 7"| \\
\verb|+     age = 21 yr 8 mo }| \\
\begin{tabular}{|r|l|r|r|r|}
\hline
\multicolumn{1}{|c}{\bf ID} &
\multicolumn{1}{|c}{\bf name} &
\multicolumn{1}{|c}{\bf weight} &
\multicolumn{1}{|c}{\bf height} &
\multicolumn{1}{|c|}{\bf age} \\
\hline
\tt @1000001 & Jill & \tt 110 lb & \tt 5' 7" & \tt 21 yr 8 mo \\
\hline
\end{tabular} \\
\verb|> all persons| \\
\begin{tabular}{|r|l|r|r|r|}
\hline
\multicolumn{1}{|c}{\bf ID} &
\multicolumn{1}{|c}{\bf name} &
\multicolumn{1}{|c}{\bf weight} &
\multicolumn{1}{|c}{\bf height} &
\multicolumn{1}{|c|}{\bf age} \\
\hline
\tt @1000000 & Jack & \tt 123 lb & \tt 5' 9" & \tt 23 yr 2 mo \\
\tt @1000001 & Jill & \tt 110 lb & \tt 5' 7" & \tt 21 yr 8 mo \\
\hline
\end{tabular} \\
\verb|> the person Jack| \\
\begin{tabular}{|r|l|r|r|r|}
\hline
\multicolumn{1}{|c}{\bf ID} &
\multicolumn{1}{|c}{\bf name} &
\multicolumn{1}{|c}{\bf weight} &
\multicolumn{1}{|c}{\bf height} &
\multicolumn{1}{|c|}{\bf age} \\
\hline
\tt @1000000 & Jack & \tt 123 lb & \tt 5' 9" & \tt 23 yr 2 mo \\
\hline
\end{tabular} \\
\verb|> the person named Jack's height| \\
\verb|5' 9"| \\
\verb|> the weight of the person named Jack| \\
\verb|123 lb| \\
\verb|> @1000001| \\
\begin{tabular}{|r|l|r|r|r|}
\hline
\multicolumn{1}{|c}{\bf ID} &
\multicolumn{1}{|c}{\bf name} &
\multicolumn{1}{|c}{\bf weight} &
\multicolumn{1}{|c}{\bf height} &
\multicolumn{1}{|c|}{\bf age} \\
\hline
\tt @1000001 & Jill & \tt 110 lb & \tt 5' 7" & \tt 21 yr 8 mo \\
\hline
\end{tabular} \\
\verb|> the weight of @1000001| \\
\verb|110 lb|
\end{indpar}

It is possible to add code to a type such as `{\tt person}':

\begin{indpar}
\verb|> a person <-- {| \\
\verb|+     body-mass-index = 703.06958 * weight in lbs| \\
\verb|+                     / (height in inches)^2 }| \\
\verb|> all persons| \\
\begin{tabular}{|r|l|r|r|r|r|}
\hline
\multicolumn{1}{|c}{\bf ID} &
\multicolumn{1}{|c}{\bf name} &
\multicolumn{1}{|c}{\bf weight} &
\multicolumn{1}{|c}{\bf height} &
\multicolumn{1}{|c|}{\bf age} &
\multicolumn{1}{|c|}{\bf \shortstack{body-mass-\\index}} \\
\hline
\tt @1000000 & Jack & \tt 123 lb & \tt 5' 9" & \tt 23 yr 2 mo & \tt 18.163738 \\
\tt @1000001 & Jill & \tt 110 lb & \tt 5' 7" & \tt 21 yr 8 mo & \tt 17.228258 \\
\hline
\end{tabular}
\end{indpar}

One can use definitions to define expressions that compute values:

\begin{indpar}
\verb|> sum from X through Y <-- integer X, integer Y {| \\
\verb|+     `Sum of integers from X through Y.'| \\
\verb|+     if ( X > Y ):| \\
\verb|+         value = 0| \\
\verb|+     else:| \\
\verb|+         value = X + sum (X+1) through Y }| \\
\verb|> sum from 5 through 10| \\
\verb|45| \\
\verb|> all (sums from X through Y)| \\
\begin{tabular}{|r|r|r|r|}
\hline
\multicolumn{1}{|c}{\bf ID} &
\multicolumn{1}{|c}{\bf X} &
\multicolumn{1}{|c}{\bf Y} &
\multicolumn{1}{|c|}{\bf value} \\
\hline
\tt @1000002 & \tt 5	& \tt	10	& \tt 45 \\
\tt @1000003 & \tt 6	& \tt	10	& \tt 40 \\
\tt @1000004 & \tt 7	& \tt	10	& \tt 34 \\
\tt @1000005 & \tt 8	& \tt	10	& \tt 27 \\
\tt @1000006 & \tt 9	& \tt	10	& \tt 19 \\
\tt @1000007 & \tt 10	& \tt	10	& \tt 10 \\
\tt @1000008 & \tt 11	& \tt	10	& \tt 0 \\
\hline
\end{tabular} \\
\verb|> sum from 1 through 2| \\
\verb|3| \\
\verb|> all (sums from X through Y)| \\
\begin{tabular}{|r|r|r|r|}
\hline
\multicolumn{1}{|c}{\bf ID} &
\multicolumn{1}{|c}{\bf X} &
\multicolumn{1}{|c}{\bf Y} &
\multicolumn{1}{|c|}{\bf value} \\
\hline
\tt @1000002 & \tt 5	& \tt	10	& \tt 45 \\
\tt @1000003 & \tt 6	& \tt	10	& \tt 40 \\
\tt @1000004 & \tt 7	& \tt	10	& \tt 34 \\
\tt @1000005 & \tt 8	& \tt	10	& \tt 27 \\
\tt @1000006 & \tt 9	& \tt	10	& \tt 19 \\
\tt @1000007 & \tt 10	& \tt	10	& \tt 10 \\
\tt @1000008 & \tt 11	& \tt	10	& \tt 0 \\
\tt @1000009 & \tt 1	& \tt	2	& \tt 3 \\
\tt @1000010 & \tt 2	& \tt	2	& \tt 2 \\
\tt @1000011 & \tt 3	& \tt	2	& \tt 0 \\
\hline
\end{tabular}
\end{indpar}

Executions can be examined in detail because
when an expression is computed, the block that computes it is remembered
for some time.  However, as memory is finite, eventually these
computations are forgotten as they can always be regenerated if
needed.

The sum above was computed by recursion: to compute the sum of 5 through 10
one computes the sum of 6 through 10.  One can also compute sums by
iteration.

\begin{indpar}
\verb|> sum from X through Y <-- integer X, integer Y {| \\
\verb|+     `Sum of integers from X through Y.'| \\
\verb|+     first sum = 0| \\
\verb|+     if ( X <= Y ):| \\
\verb|+         next sum = sum + X| \\
\verb|+         next X = X + 1| \\
\verb|+     else:| \\
\verb|+         value = sum }| \\
\verb|> sum from 5 through 10| \\
\verb|45| \\
\verb|> all (sums from X through Y)| \\
\begin{tabular}{|r|r|r|r|r|r|}
\hline
\multicolumn{1}{|c}{\bf ID} &
\multicolumn{1}{|c}{\bf previous} &
\multicolumn{1}{|c}{\bf X} &
\multicolumn{1}{|c}{\bf Y} &
\multicolumn{1}{|c}{\bf sum} &
\multicolumn{1}{|c|}{\bf value} \\
\hline
\tt @1000012 &              & \tt 5	& \tt 10   & \tt 0	& \\
\tt @1000013 & \tt @1000012 & \tt 6	& \tt 10   & \tt 5	& \\
\tt @1000014 & \tt @1000013 & \tt 7	& \tt 10   & \tt 11	& \\
\tt @1000015 & \tt @1000014 & \tt 8	& \tt 10   & \tt 18	& \\
\tt @1000016 & \tt @1000015 & \tt 9	& \tt 10   & \tt 26	& \\
\tt @1000017 & \tt @1000016 & \tt 10	& \tt 10   & \tt 35	& \\
\tt @1000018 & \tt @1000017 & \tt 11	& \tt 10   & \tt 45	& \tt 45 \\
\hline
\end{tabular}
\end{indpar}

One can also iterate over data.

\begin{indpar}
\verb|> average weight of X <-- tuple X of persons {| \\
\verb|+     first count = 0| \\
\verb|+     first sum = 0| \\
\verb|+     if X = ():| \\
\verb|+         if count = 0:| \\
\verb|+             value = error `Cannot average 0 things.'| \\
\verb|+         else:| \\
\verb|+             value = sum / count| \\
\verb|+     else:| \\
\verb|+         next count = count + 1| \\
\verb|+         next sum = sum + the weight of (first X)| \\
\verb|+         next X = rest X }| \\
\verb|> average weight of (all persons)| \\
\verb|116.5 lbs|
\end{indpar}

In case you wonder how some of the above works, here are some hints.

PCASL tends to ignore word endings: thus `{\tt person}' and
`{\tt persons}' are to PCASL the same word.
PCASL can even be told that `{\tt person}'
and `{\tt people}' are the same word.
`{\tt Jack's}', on the other hand, is treated an abbreviation
of two separate words `{\tt Jack}' and `{\tt 's}', where `{\tt 's}' is
a separate word by itself.

Expressions are just strings of words and subexpressions.  Subexpressions
must be parenthesized unless they consist of a single word, or unless
they are delimited by operators.

Lists of values can be stored in tuples, which are computed by comma
separated lists in parentheses.  Thus

\begin{indpar}
\verb|> (the person named Jill, the person named Jack)| \\
\begin{tabular}{|r|l|r|r|r|}
\hline
\multicolumn{1}{|c}{\bf ID} &
\multicolumn{1}{|c}{\bf name} &
\multicolumn{1}{|c}{\bf weight} &
\multicolumn{1}{|c}{\bf height} &
\multicolumn{1}{|c|}{\bf age} \\
\hline
\tt @1000001 & Jill & \tt 110 lb & \tt 5' 7" & \tt 21 yr 8 mo \\
\tt @1000000 & Jack & \tt 123 lb & \tt 5' 9" & \tt 23 yr 2 mo \\
\hline
\end{tabular} \\
\verb|> (the person named Jack, the person named Jill)| \\
\begin{tabular}{|r|l|r|r|r|}
\hline
\multicolumn{1}{|c}{\bf ID} &
\multicolumn{1}{|c}{\bf name} &
\multicolumn{1}{|c}{\bf weight} &
\multicolumn{1}{|c}{\bf height} &
\multicolumn{1}{|c|}{\bf age} \\
\hline
\tt @1000000 & Jack & \tt 123 lb & \tt 5' 9" & \tt 23 yr 2 mo \\
\tt @1000001 & Jill & \tt 110 lb & \tt 5' 7" & \tt 21 yr 8 mo \\
\hline
\end{tabular} \\
\verb|> raw (all persons)| \\
\verb|(the person named Jack, the person named Jill)|
\end{indpar}

The `{\tt raw}' form of a value is a form that prints as you could
input it in a way that reveals its internal structure.  Thus
`{\tt all persons}' denotes the tuple of all persons.

`{\tt the person named Jack}' is a printed representation of the
internal name of a block.  Such printed representations are chosen
automatically from the set of all possible representations, which
in this case include `{\tt the person weighing 123 lbs}' and
`{\tt the person named Jack weighing 123 lbs}'.

A single non-tuple value is equivalent to a tuple with one element.
Tuples cannot have other tuples as elements; instead attempts to
compute such tuples are \key{flattened}:

\begin{indpar}
\verb|> x = (1,(2,3),4)| \\
\verb|(1,2,3,4)| \\
\verb|> first x| \\
\verb|1| \\
\verb|> rest x| \\
\verb|(2,3,4)| \\
\verb|> rest (rest x)| \\
\verb|(3,4)| \\
\verb|> rest (rest (rest x))| \\
\verb|4| \\
\verb|> rest 4| \\
\verb|()|
\end{indpar}

PCASL has many different kinds of quotes or brackets.  Some of these,
\verb|{|\ldots\verb|}|, \verb|`|\ldots\verb|'|, and \verb|{{|\ldots\verb|}}|,
turn evaluation off, while \verb|<<|\ldots\verb|>>|,
\verb|<<<|\ldots\verb|>>>|, \verb|<<<<|\ldots\verb|>>>>|, etc.
turn evaluation on.
Some,
\verb|`|\ldots\verb|'| and \verb|{{|\ldots\verb|}}|,
turn recognition of operators (e.g., \verb|+| and \verb|=|) off, while others,
\verb|<<|\ldots\verb|>>| and \verb|{|\ldots\verb|}|,
turn recognition of operators on.
\verb|`|\ldots\verb|'|,
\verb|``|\ldots\verb|''|, \verb|```|\ldots\verb|'''|, etc.
also do other things, like insert implicit operations (e.g., \verb|sentence|).

PCASL stores information as expressions.  For example:

\begin{indpar}
\verb|> (a person named `Jack') is husband of (a person named `Jill') <--| \\
\verb|> Y is wife of X <-- X is husband of Y| \\
\verb|> (a person named `Jill') is wife of (a person named `Jack') ?| \\
\verb|true| \\
\verb|> (a person named `Jack') is wife of (a person named `Jill') ?| \\
\verb|false| \\
\verb|> (a person named X) is wife of (a person named `Jack') ?| \\
\verb|X = `|Jill\verb|'| \\
\verb|> X is wife of (a person named `Jack') ?| \\
\verb|X = (a person named `Jill')| \\
\verb|> @1000001 is wife of (a person named `Jack') ?| \\
\verb|true| \\
\verb|> @1000001 = (a person named `Jill') ?| \\
\verb|true| \\
\verb|> @1000001 = (a person named `Jack') ?| \\
\verb|false|
\end{indpar}

PCASL also supports pictorial data
that are computed like expressions.

\begin{indpar}
\verb|> x = {circle 0.4}| \\
\begin{picture}(0.4,0.4)
\put(0.2,0.2){\circle{0.4}}
\end{picture} \\
\verb|> y = {rectangle (0.4,0.2)}| \\
\begin{picture}(0.4,0.2)
\put(0,0){\framebox(0.4,0.2){}}
\end{picture} \\
\verb|> z = {(circle 0.4) labeled `Jack'}| \\
\begin{picture}(0.4,0.4)
\put(0.2,0.2){\circle{0.4}}
\put(0.0,0.0){\makebox(0.4,0.4){Jack}}
\end{picture} \\
\verb|> {(<<x>> right of <<y>>) above <<z>>}| \\
\begin{picture}(0.8,0.8)
\put(0.6,0.6){\circle{0.4}}
\put(0.0,0.5){\framebox(0.4,0.2){}}
\put(0.4,0.2){\circle{0.4}}
\put(0.2,0.0){\makebox(0.4,0.4){Jack}}
\end{picture} \\
\verb|> {row(<<x>>,<<y>>,<<z>>)}| \\
\begin{picture}(1.4,0.4)
\put(0.2,0.2){\circle{0.4}}
\put(0.5,0.1){\framebox(0.4,0.2){}}
\put(1.2,0.2){\circle{0.4}}
\put(1.0,0.0){\makebox(0.4,0.4){Jack}}
\end{picture} \\
\verb|> p = {column (row(<<x>>,<<y>>,<<z>>), row(<<z>>,<<y>>,<<x>>))}| \\
\begin{picture}(1.4,0.9)
\put(0.2,0.7){\circle{0.4}}
\put(0.5,0.6){\framebox(0.4,0.2){}}
\put(1.2,0.7){\circle{0.4}}
\put(1.0,0.5){\makebox(0.4,0.4){Jack}}
\put(1.2,0.2){\circle{0.4}}
\put(0.5,0.1){\framebox(0.4,0.2){}}
\put(0.2,0.2){\circle{0.4}}
\put(0.0,0.0){\makebox(0.4,0.4){Jack}}
\end{picture} \\
\verb|> raw x| \\
\verb|{circle 0.4}| \\
\verb|> raw p| \\
\verb|{column (row (circle 0.4,| \\
\verb|              rectangle (0.4,0.2),| \\
\verb|              (circle 0.4) labeled `Jack'),| \\
\verb|         row ((circle 0.4) labeled `Jack',| \\
\verb|              rectangle (0.4,0.2),| \\
\verb|              circle 0.4))}|
\end{indpar}

You can also change how an expression is displayed.

\begin{indpar}
\newlength{\ovalraise}
\setlength{\ovalraise}{-0.1in}
\addtolength{\ovalraise}{0.8ex}
\verb|> display ( P ) <-- person ( P ) has name ( X ) {| \\
\verb|>     value = {oval (0.4,0.2) labeled <<X>>} }| \\
\verb|> (a person named `Jack')| \\
\begin{picture}(0.4,0.2)
\put(0.2,0.1){\oval(0.4,0.2)}
\put(0.0,0.0){\makebox(0.4,0.2){Jack}}
\end{picture} \\
\verb|> ``(a person named `Jill') is wife of (a person named `Jack')''| \\
\verb|``|
\raisebox{\ovalraise}{\begin{picture}(0.4,0.2)
\put(0.2,0.1){\oval(0.4,0.2)}
\put(0.0,0.0){\makebox(0.4,0.2){Jill}}
\end{picture}}
{is the wife of}
\raisebox{\ovalraise}{\begin{picture}(0.4,0.2)
\put(0.2,0.1){\oval(0.4,0.2)}
\put(0.0,0.0){\makebox(0.4,0.2){Jack}}
\end{picture}}
\verb|''|
\end{indpar}

Displays can be used to make demonstrations:

\begin{indpar}
\verb|> a demo {| \\
\verb|>     on a demo with angle X <-- {| \\
\verb|+         angle = X } }| \\
\verb|> x = a demo with angle 30 degrees| \\
\begin{tabular}{|r|r|}
\hline
\multicolumn{1}{|c}{\bf ID} &
\multicolumn{1}{|c|}{\bf value} \\
\hline
\tt @1000043 & 30 degrees \\
\hline
\end{tabular} \\
\verb|> a demo {| \\
\verb|+     on update THIS to X <-- {| \\
\verb|+         next angle = X }| \\
\verb|+     on increment THIS by X <-- {| \\
\verb|+         next angle = angle + X } }| \\
\verb|> update x to 40 degrees| \\
\begin{tabular}{|r|r|}
\hline
\multicolumn{1}{|c}{\bf ID} &
\multicolumn{1}{|c|}{\bf value} \\
\hline
\tt @1000044 & 40 degrees \\
\hline
\end{tabular} \\
\verb|> increment x by 5 degrees| \\
\begin{tabular}{|r|r|}
\hline
\multicolumn{1}{|c}{\bf ID} &
\multicolumn{1}{|c|}{\bf value} \\
\hline
\tt @1000045 & 45 degrees \\
\hline
\end{tabular} \\
\verb|> display ( D ) <-- demo ( D ) with angle ( X ) {| \\
\verb|+     c = {circle 1.0 dotted center (0.0,0.0)}| \\
\verb|+     x-axis = {arrow from (-0.75,0.0) to (0.75,0.0)}| \\
\verb|+     y-axis = {arrow from (0.0,-0.75) to (0.0,0.75)}| \\
\verb|+     line = {line from (0.0,0.0) to <<(0.5*cos X, 0.5*sin X)>>}| \\
\verb|+     arc = {arc-arrow from (0.7,0.0) to <<(0.3*cos X, 0.3*sin X)>>}| \\
\verb|+     theta = {Greek th}| \\
\verb|+     value = {column (| \\
\verb|+                overlap (| \\
\verb|+                  <<c>>,| \\
\verb|+                  <<x-axis>> labeled `X Axis',| \\
\verb|+                  <<y-axis>> labeled `Y Axis',| \\
\verb|+                  <<line>>,| \\
\verb|+                  <<arc>> labeled `<<theta>>' ),| \\
\verb|+                label `Depiction of Angle <<theta>>' ) } } | \\
\verb|> x| \\
\begin{picture}(4.0,1.9)
\put(1.0,0.95){\qbezier[40](0.5,0.0)(0.46,0.46)(0.0,0.5)}
\put(1.0,0.95){\qbezier[40](0.0,0.5)(-0.46,0.46)(-0.5,0.0)}
\put(1.0,0.95){\qbezier[40](-0.5,0.0)(-0.46,-0.46)(0.0,-0.5)}
\put(1.0,0.95){\qbezier[40](0.0,-0.5)(0.46,-0.46)(0.5,0.0)}
\put(1.0,0.95){
    \qbezier[250](0.3,0.0)(0.3,0.124264)(0.212132,0.212132)
    \put(0.212132,0.212132){\vector(-1,1){0.00}}
    \put(0.31,0.1){$\theta$}}
\put(1.0,0.2){\vector(0,1){1.5}}
\put(0.0,1.7){\makebox(2.0,0.2){Y Axis}}
\put(0.25,0.95){\vector(1,0){1.5}}
\put(1.80,0.85){\makebox(2.0,0.2)[l]{X Axis}}
\put(1.0,0.95){\line(1,1){0.353553}}
\put(0.0,0.0){\makebox(2.0,0.2){Depiction of Angle $\theta$}}
\end{picture} \\
\verb|> show x label {Greek th}| \\
See $\theta$ \\
\verb|> update x to 40 degrees| \\
See $\theta$ \\
\verb|> increment x by -5 degrees| \\
See $\theta$
\end{indpar} 

In this example we first define a `constructor'
of the form `{\tt a demo with angle X}' to make new {\tt demo} blocks,
and then we define two `methods', namely `{\tt update THIS to X}' and
`{\tt in\-cre\-ment THIS by X}', to change a {\tt demo} block.  Changing
a {\tt demo} block is like iterating a loop to make a new block.

Next we define how to display a {\tt demo} block.  Then we use the
`{\tt show x label {\ttbrackets \{Greek th\}}}' command
to cause the {\tt demo} block value of {\tt x}
to be displayed in a separate window labeled `$\theta$'.
Every time this {\tt demo} block
changes, the $\theta$ window is updated, and every time the block
is to be printed, `See $\theta$' is printed instead.


TBD: example of a simple game.

The rest of this document is a reference manual for PCASL.


\section{Lexical Scans}

A PCASL program is a sequence of characters which is scanned from
left to right to produce a sequence of pre-lexemes.  The sequence
of pre-lexemes is then scanned from left to right to produce
a sequence of lexemes.  The sequence of lexemes is then subject to
a left to right lexical parsing scan to produce a final set of lexemes
that is input for expression parsing.

For example, the input `{\tt x = 7' 25";}' contains the pre-lexeme
`{\tt 5.5";}', which is split into 3 lexemes `{\tt 5.5 " ;}', and
then during lexical parsing numbers and units are grouped
using parentheses to produce the following string of 9 lexemes:
`{\tt x = ( 7 ' 25 " ) ;}'.

\subsection{Pre-Lexemes}

Pre-lexemes are defined as follows:

\begin{indpar}
\key{pre-lexeme}
	\begin{tabular}[t]{rl}
	::= & \key{pre-word} \\
	$|$ & \key{opening-mark} \\
	$|$ & \key{closing-mark} \\
	$|$ & \key{format-separator} \\
	$|$ & \key{white-space}
	\end{tabular}
	\\[1ex]
\key{pre-word} ::= {\em word-character} {\em word-character}$^\star$ \\[1ex]
\key{word-character} \begin{tabular}[t]{rl}
                       ::= & {\em letter} $|$ {\em digit} \\
		       $|$ &    \verb|+|%
		             $|$\verb|-|%
		             $|$\verb|*|%
		             $|$\verb|/|%
		             $|$\verb|\|%
		             $|$\verb|~|%
		             $|$\verb|@|%
		             $|$\verb|#|%
		             $|$\verb|$|%
		             $|$\verb|%|%
		             $|$\verb|^|%
		             $|$\verb|&|%
		             $|$\verb|=|%
		             $|$\verb/|/%
		             $|$\verb|<|%
		             $|$\verb|>|%
		             $|$\verb|_|%
		             $|$\verb|"|%
		             $|$\verb|`|%
		             $|$\verb|'|%
		             $|$\verb|!|%
		             $|$\verb|?|%
			     $|$\verb|;|%
			     $|$\verb|:|%
			     $|$\verb|,|%
		             $|$\verb|.|
			\end{tabular}\\[1ex]
\key{letter} ::= {\em lower-case-letter} $|$ {\em upper-case-letter} \\[1ex]
\key{lower-case-letter} ::=    \verb|a|%
			    $|$\verb|b|%
			    $|$\verb|c|%
			    $|$\verb|d|%
			    $|$\verb|e|%
			    $|$\verb|f|%
			    $|$\verb|g|%
			    $|$\verb|h|%
			    $|$\verb|i|%
			    $|$\verb|j|%
			    $|$\verb|k|%
			    $|$\verb|l|%
			    $|$\verb|m|%
			    $|$\verb|n|%
			    $|$\verb|o|%
			    $|$\verb|p|%
			    $|$\verb|q|%
			    $|$\verb|r|%
			    $|$\verb|s|%
			    $|$\verb|t|%
			    $|$\verb|u|%
			    $|$\verb|v|%
			    $|$\verb|w|%
			    $|$\verb|x|%
			    $|$\verb|y|%
			    $|$\verb|z|
			    \\[1ex]
\key{upper-case-letter} ::=    \verb|A|%
			    $|$\verb|B|%
			    $|$\verb|C|%
			    $|$\verb|D|%
			    $|$\verb|E|%
			    $|$\verb|F|%
			    $|$\verb|G|%
			    $|$\verb|H|%
			    $|$\verb|I|%
			    $|$\verb|J|%
			    $|$\verb|K|%
			    $|$\verb|L|%
			    $|$\verb|M|%
			    $|$\verb|N|%
			    $|$\verb|O|%
			    $|$\verb|p|%
			    $|$\verb|Q|%
			    $|$\verb|R|%
			    $|$\verb|S|%
			    $|$\verb|T|%
			    $|$\verb|U|%
			    $|$\verb|V|%
			    $|$\verb|W|%
			    $|$\verb|X|%
			    $|$\verb|Y|%
			    $|$\verb|Z|
			    \\[1ex]
\key{digit} ::=    \verb|0|%
		$|$\verb|1|%
		$|$\verb|2|%
		$|$\verb|3|%
		$|$\verb|4|%
		$|$\verb|5|%
		$|$\verb|6|%
		$|$\verb|7|%
		$|$\verb|8|%
		$|$\verb|9|
		\\[1ex]
\key{opening-mark} ::=     \verb|(|
	       	       $|$ \verb|[|
	       	       $|$ \verb|{|
	       	       $|$ {\em opening-quote}
	       	       $|$ {\em opening-angle}
	       	       \\[1ex]
\key{closing-mark} ::=     \verb|)|
	       	       $|$ \verb|]|
	       	       $|$ \verb|}|
	       	       $|$ {\em closing-quote}
	       	       $|$ {\em closing-angle}
	       	       \\[1ex]
\key{opening-quote} \begin{tabular}[t]{rl}
		     ::= & {\em opening-quote-character}
		           {\em opening-quote-character}$^\star$ \\
		     \end{tabular}
		     \\[1ex]
\key{opening-quote-character} ::= \verb|`| \\[1ex]
\key{closing-quote} \begin{tabular}[t]{rl}
		     ::= & {\em closing-quote-character}
		           {\em closing-quote-character}$^\star$ \\
		     \end{tabular}
		     \\[1ex]
\key{closing-quote-character} ::= \verb|'| \\[1ex]
\key{quote} ::= {\em opening-quote} $|$ {\em closing-quote} \\[1ex]
\key{opening-angle} \begin{tabular}[t]{rl}
		     ::= & {\em opening-angle-character}
		           {\em opening-angle-character} \\
		         & {\em opening-angle-character}$^\star$ \\
		     \end{tabular}
		     \\[1ex]
\key{opening-angle-character} ::= \verb|<| \\[1ex]
\key{closing-angle} \begin{tabular}[t]{rl}
		     ::= & {\em closing-angle-character}
		           {\em closing-angle-character} \\
		         & {\em closing-angle-character}$^\star$ \\
		     \end{tabular}
		     \\[1ex]
\key{closing-angle-character} ::= \verb|>| \\[1ex]
\key{angle-character} ::= \verb|<| $|$ \verb|>| \\[1ex]
\key{opening-character} ::=     \verb|(|
		            $|$ \verb|[|
		            $|$ \verb|{|
		            $|$ \verb|`|
		            $|$ \verb|<|
			    \\[1ex]
\key{closing-character} ::=     \verb|)|
		            $|$ \verb|]|
		            $|$ \verb|}|
		            $|$ \verb|'|
		            $|$ \verb|>|
			    \\[1ex]
\key{format-separator} ::= {\em format-separator-character}
			   {\em format-separator-character}$^\star$ \\[1ex]
\key{format-separator-character} ::=    \verb\|\
				        \\[1ex]
\key{white-space} ::= {\em white-space-character}
                     {\em white-space-character}$^\star$ \\[1ex]
\key{white-space-character} ::=
    \key{horizontal-space-character} $|$ \key{vertical-space-character} \\[1ex]
\key{horizontal-space-character} ::=
    \key{space} $|$ \key{horizontal-tab} $|$ \key{carriage-return} \\[1ex]
\key{vertical-space-character} ::=
    \key{line-feed} $|$ \key{vertical-tab} $|$ \key{form-feed}
\end{indpar}


The following sections give rules involving pre-lexemes.

\subsubsection{Lexical Matching and Pre-Lexical Context}
\label{LEXICAL-MATCHING}

{\em Opening-marks} and {\em closing-marks} are both pre-lexemes
and also lexemes.  Rules for matching pre-lexemes in a pre-lexeme
sequence are the same as rules for matching lexemes in a lexeme sequence.
Here we will state the rules for lexemes, and leave it to the reader
to reformulate them for pre-lexemes.

{\bf Lexeme Matching Rule.}\index{Lexeme Matching Rule}
An {\em opening-mark} with $N$ characters $C$
must have a matching {\em closing-mark} with $N$ characters each the
mirror of $C$.  Here
the mirror of \verb|`| is \verb|'|,
the mirror of \verb|{| is \verb|}|,
the mirror of \verb|[| is \verb|]|,
the mirror of \verb|(| is \verb|)|,
and the mirror of \verb|<| is \verb|>|.
Each {\em closing-mark} must match exactly one {\em opening-mark},
each {\em opening-mark} must match exactly one {\em closing-mark},
and an {\em opening-mark} must precede its matching {\em closing-mark}.

Two lexemes are said to be \mkey{matched}{lexemes} if and only if
they are matched opening and closing marks.

{\bf Matched Lexeme Nesting Rule.}\index{Matched Lexeme Nesting Rule}
If one lexeme in a pair $P_2$ of matched lexemes is in between the
lexemes of another pair $P_1$ of matched lexemes, then both lexemes
in $P_2$ must be in between the lexemes of $P_1$.  In this case
$P_2$ is said to be \mkey{nested}{lexemes} inside of $P_1$.

During the scan a character $C$ is said to be in the
\key{lexical context} of a pair $P$ of matched lexemes
if and only if $C$ is between the matched lexemes of $P$,
and $C$ is not between any other pair of matched
lexemes that is nested inside of $P$.  For example,
in \verb|{ x [ y ] z }|, \verb|{ }| are in the outermost pre-lexical
context, \verb|x [ ] z| are in the middle pre-lexical context whose
matched lexemes are \verb|{ }|, and
\verb|y| is in the innermost pre-lexical context whose matched lexemes
are \verb|[ ]|.

For a sequence of pre-lexemes, \mkey{matched}{pre-lexemes} pre-lexemes
and \key{pre-lexical context} are defined as for a sequence of lexemes.

\subsubsection{Character Disambiguation Rules}

Several characters in pre-lexemes are ambiguous in the
pre-lexeme syntax equations.  The following rules disambiguate these
characters.

{\bf Opening Quote Rule.}\index{Opening Quote Rule}
An \key{opening-quote-character} must be preceded by a
{\em white-space-character},
an {\em opening-character}, or a {\em format-separator-character}.
Otherwise it is a {\em word-character}.

{\bf Closing Quote Rule.}\index{Closing Quote Rule}
A \key{closing-quote-character} must be part of a sequence of
{\em closing-quote-characters} that is of exactly the right length to be the
matching pre-lexeme for the last previous unmatched {\em opening-mark},
which must be an {\em opening-quote}.
Otherwise the potential {\em closing-quote-character}
is a {\em word-character}.

{\bf Format Separator Rule.}\index{Format Separator Rule}
A {\em format-separator-character} must be in the pre-lexical context of
a pair of matched {\em quotes}.  Otherwise it is a {\em word-character}.

{\bf Angle Rule.}\index{Angle Rule}
An {\em angle-character} must be either preceded by or followed by a copy
of itself.  Otherwise it is a {\em word-character}.

\subsubsection{Pre-Lexeme Examples}

Quote Examples:

\begin{indpar}[0.1in]
\begin{tabular}{@{}l@{\hspace*{0.3in}}l}
			& Pre-Lexeme Sequence		\\
Input String		& with {\em spaces} represented by \verb|_|'s \\[1ex]
\verb|I said `Hello'.|	& \tt I~~\verb|_|~~said~~\verb|_|~~`~~Hello~~'~~.	\\
\verb|Re`op 'tis.  But!|
			& \tt Re`op~~\verb|_|~~'tis.%
			  ~~\verb|__|~~But! \\
\verb|`Like 'tis'.|	& \tt `~~Like~~\verb|_|~~'~~tis'.	\\
\verb|``Like 'tis''.|	& \tt ``~~Like~~\verb|_|~~'tis~~''~~.	\\
\verb|`Like me''.|	& \tt `~~Like~~\verb|_|~~me''.		\\
\verb|`` `Hello' is a word.''|
			& \tt ``~~\verb|_|~~`~~Hello~~'~~\verb|_|~~is%
			  ~~\verb|_|~~a~~\verb|_|~~word.~~'' \\
\end{tabular}
\end{indpar}

Other rules of PCASL (\secref{LEXEMES})
limit the semantic content
of {\em white-space}, so that there is no problem putting
space between the \verb|``| and \verb|`| in the last example.

Angle and Format Separator Examples:

\begin{indpar}[0.1in]
\begin{tabular}{@{}l@{\hspace*{0.6in}}l}
			& Pre-Lexeme Sequence		\\
Input String		& with {\em spaces} represented by \verb|_|'s \\[1ex]
\verb|x <= y |		& \tt x \verb|_| \verb|<=| \verb|_| y	\\
\verb|x <<= y |		& \tt x \verb|_| \verb|<<| = \verb|_| y	\\
\verb/x = y|z/		& \tt x \verb|_| = \verb|_| \verb/y|z/ \\
\verb/x = `y|z'/	& \tt x \verb|_| = \verb|_|
                              \verb|`| y \verb/|/ z \verb|'| \\
\end{tabular}
\end{indpar}

\subsubsection{White Space Conversion}
\label{WHITE-SPACE-CONVERSION}

A \key{white-space} pre-lexeme does not have exactly the same characters
that were input to create it, unlike other pre-lexemes.  The
sequence of {\em white-space-characters} input to create a
{\em white-space} pre-lexeme is modified as follows to create the pre-lexeme:

{\bf Line End Spaces Rule.}\index{Line End Spaces Rule}
All {\em horizontal-space} characters preceding a {\em vertical-space}
character are deleted.  Thus spaces at line ends are ignored.

{\bf Carriage Return Rule.}\index{Carriage Return Rule}
If the pre-lexeme contains a {\em carriage-return} but no
{\em vertical-space-character}, it is in error.
Each {\em carriage-return} and all {\em horizontal-spaces} preceding
it are deleted.  Thus the pre-lexeme has no {\em carriage-returns}.

{\bf Horizontal Tab Rule.}\index{Horizontal Tab Rule}
Each {\em horizontal-tab} is replace by {\em spaces} assuming that
horizontal tab stops are set every 8 columns.
Thus the pre-lexeme has no {\em horizontal-tabs}.

Note that these rules do \underline{not} alter the printed appearance
of the {\em white-space} pre-lexeme, assuming that each
{\em vertical-space-character} causes printing to return to the beginning
of the line after the vertical space is executed.  The input may or may not
contain {\em carriage-returns} before or after {\em vertical-space-characters}
with no effect.

After these rules are applied, a {\em white-space} pre-lexeme consists
of zero or more {\em vertical-space-characters} followed by
zero or more {\em space} characters.  It is not possible to have an
empty {\em white-space} pre-lexeme; the pre-lexeme either has a
{\em vertical-space-character} which none of the above rules delete,
or it has a {\em space} character which cannot be deleted because
in the absence of a {\em vertical-space-character} the input
to the pre-lexeme cannot have a {\em carriage-return}.

{\em White-space} pre-lexemes in certain pre-lexical contexts
are used to create indentation lexemes
when pre-lexemes are converted to lexemes (\secref{INDENTATION-LEXEMES}).
{\em White-space} pre-lexemes in the pre-lexical context of
a pair of matched {\em quotes} become lexemes.
All other {\em white-space} pre-lexemes are ignored when
pre-lexemes are converted to lexemes.  All {\em white-space} pre-lexemes
that are not in the pre-lexical context of a pair of matched {\em quotes}
are discarded and do not become lexemes.

The rules for forming lexemes from pre-lexemes ensure that all
non-empty sequences of {\em vertical-space} characters are equivalent,
except that within the pre-lexical context of a pair of matched {\em quotes}
a distinction is made between a {\em white-space} lexeme that contains
only a single {\em line-feed}, and represents just the end of a non-blank line,
and a {\em white-space} lexeme that represents one or more blank lines, because
it contains two or more {\em line-feeds},
or because it contains
a {\em vertical-space} character that is not a {\em line-feed}.


\subsection{Lexemes}
\label{LEXEMES}

The sequence of pre-lexemes is converted to a sequence of lexemes
according to the following rules, which we will describe in the
order given:

\begin{indpar}[1in]
Numbers \\
Post Separators \\
Indentation Lexemes
\end{indpar}

The syntax equations defining a lexeme are:

\begin{indpar}
\key{lexeme} ::= {\em word} $|$ {\em separator}
		$|$ {\em opening-mark}
		$|$ {\em closing-mark}
		$|$ {\em white-space} \\[1ex]
\key{separator} ::= {\em format-separator}
		$|$ {\em post-separator} \\[1ex]
\key{indentation-lexeme} ::= \verb|;|
			 $|$ \verb|{|
			 $|$ \verb|}|
\end{indpar}

{\em Post-separators} are single characters removed from the
ends of {\em pre-words}.  A {\em word} is what is left of
a {\em pre-word} after any {\em post-separators}
are removed from its end.
{\em Words} and {\em post-separators} are defined
below in \Secref{POST-SEPARATORS}.


The {\em indentation-lexemes} are lexemes implied by indentation, and
are not distinguishable from explicit lexemes, namely the post-separator
\verb|;|, the {\em opening-mark} \verb|{|, and the {\em closing-mark}
\verb|}|.  They are defined below in
\secref{INDENTATION-LEXEMES}.

{\em White-space} pre-lexemes become lexemes if they appear
in the pre-lexical context of a pair of matched {\em quotes}.
All other {\em white-space} pre-lexemes are discarded when pre-lexemes
are scanned to lexemes.

{\em Format-separators}, {\em opening-marks}, and {\em closing-marks}
are both pre-lexemes and lexemes.

\subsubsection{Number Lexemes}
\label{NUMBER-LEXEMES}

The rule for splitting a {\em pre-word} into a {\em word}
and a {\em separator} makes reference to {\em words} that are
{\em numbers}, saying that splitting is preferred if the {\em word}
resulting from the split is a {\em number}.  Some of the syntax equations
defining {\em numbers} are as follows.

\begin{indpar}
\key{number} ::= {\em real-number} $|$ {\em unit-number} \\[1ex]
\key{real-number} ::= {\em decimal-number} $|$ {\em radix-number}
				           $|$ {\em ratio}
				           $|$ {\em scientific-number}
\end{indpar}

\subsubsubsection{Decimal Numbers}
\label{DECIMAL-NUMBERS}

Decimal numbers are sequences of digits with optional commas,
an optional decimal point, and an optional sign.

\begin{indpar}
\key{decimal-number} ::= {\em unsigned-decimal-number}
                     $|$ {\em sign} {\em unsigned-decimal-number} \\[1ex]
\key{sign} ::= \verb|+| $|$ \verb|-| \\[1ex]
\key{unsigned-decimal-number} \begin{tabular}[t]{rl}
                              ::= & {\em decimal-natural} \\
		              $|$ & {\em decimal-natural} \verb|.|
		                    {\em decimal-natural} \\
		              $|$ & \verb|.| {\em decimal-natural}
		              \end{tabular} \\[1ex]
\key{decimal-natural} ::= {\em decimal-digits}
		     $|$ {\em decimal-natural} \verb|,| {\em decimal-digits}
		     \\[1ex]
\key{decimal-digits} ::= {\em digit} {\em digit}$^\star$ \\[1ex]
\key{decimal-ratio} ::= {\em decimal-natural} \verb|/|
			{\em decimal-natural}
\end{indpar}

Decimal Number Examples:

\begin{indpar}[0.1in]
\tt
\begin{tabular}{l@{~~~~~}l@{~~~~~}l@{~~~~~}l@{~~~~~}l}
123	& -123		& +123		& 1,234		& -1,234,567 \\
123.0	& -.123		& +0.0		& +.000		& 1,234.987,654 \\
1,2	& -1.86,54	& 1,234567.89	& 12345.678,9	& +1.234567,892
\end{tabular}
\end{indpar}

In a {\em decimal-number} lexeme, the decimal point must be followed by
a digit, and commas must be surrounded by digits.

In addition, commas must be located
every 3 digits from the decimal point, or every 3 digits
from the right end if there is no decimal point.
If there are any commas at all, there must be commas every 3 digits
in the integer part, while the fraction part may be comma free,
or may contain commas every 3 digits.
Failure to follow the rules of this paragraph will result in an
error when the lexeme is converted to a number,
but is \underline{not} an error of lexeme formation.  The last line of
examples above are therefore legal lexemes that will give errors when
converted to numbers.

\subsubsubsection{Radix Numbers}
\label{RADIXED-NUMBERS}

\ikey{Radix-numbers}{radix-number} permit binary, octal, or hexadecimal
radixes to the used instead of decimal.  In fact, other number representation
schemes are permitted.

\begin{indpar}
\key{radix-number} ::= {\em unsigned-radix-number}
		   $|$ {\em sign} {\em unsigned-radix-number} \\[1ex]
\key{unsigned-radix-number} ::= {\em radix-indicator}
		        \verb|#| {\em radix-number-mark}
		        {\em radix-number-mark}$^\star$ \verb|#|
		        \\[1ex]
\key{radix-indicator} ::= {\em letter} {\em letter-or-digit}$^\star$ \\[1ex]
\key{letter-or-digit} ::= {\em letter} $|$ {\em digit} \\[1ex]
\key{radix-number-mark} ::= {\em word-character} except
	\{ \verb|#| $|$ \verb/|/ $|$ \verb|<| $|$ \verb|>|
	            $|$ \verb|`| $|$ \verb|'| \}
\end{indpar}

Radix Number Examples:

\begin{indpar}[0.1in]
\tt
\begin{tabular}{l@{~~~~~}l@{~~~~~}l@{~~~~~}l}

B\#10110100\#	& O\#77534201\# & D\#19758\#	& X\#FE8A932B\# \\
B\#101101\#	& O\#0.7753\#	& D\#197.58\#	& X\#0.fe8a932b\# \\
B\#10,1101\#	& O\#12,3456\#  & D\#0.123,5\#	& X\#FE8A,932B.7CCD,83\# \\
B\#a5,7b63\#	& O\#12/3456\#	& D\#+0.1\#	& X\#F.E8A932B.7CCD\# \\
\end{tabular}
\end{indpar}


The following {\em radix-indicators} are standard.

\begin{center}
\begin{tabular}{l@{~~~~~}l@{~~~~~}l@{~~~~~}l}
Radix	& Radix		 &			& Allowed Digits \\
Name	& Indicators     & Allowed Digits	& Between Commas \\[1ex]
binary	& \tt b~~~B	 & \tt 0 1		& 4 or 8 \\
octal	& \tt o~~~O	 & \tt 0 1 2 3 4 5 6 7	& 3 \\
decimal	& \tt d~~~D	 & \tt 0 1 2 3 4 5 6 7 8 9
						& 3 \\
hexadecimal
	& \tt x~~~X	 & \tt 0 1 2 3 4 5 6 7 8 9
						& 4 or 8 \\
	&		 & \tt a b c d e f A B C D E F
\end{tabular}
\end{center}

The rules for comma location are the same as for {\em decimal-numbers}
(\secref{DECIMAL-NUMBERS}), except the number of digits between commas may
be 4 or 8 instead of 3.

{\em Radix-numbers} can be legal lexemes and still be unconvertible to
numbers because their {\em radix-indicators} are undefined, they
have {\em radix-number-marks} (e.g., digits)
not defined for the given {\em radix-indicator}, they
have too many decimal points, and so forth.
These errors are detected when the lexemes are converted to numbers.
The last line of {\em radix-number} examples above are legal lexemes that
will give errors when converted to numbers.
Note that standard radix indicators only allow commas, a decimal point,
and digits as {\em radix-number-marks}.

On the other hand, it
is possible to define non-standard converters for converting
{\em radix-number} lexemes to numbers, and thereby increase the
space of number representations.

\subsubsubsection{Ratios}
\label{RATIOS}

A {\em ratio} consists of two strings of decimal digits, the numerator
and denominator, separated by a slash (`\verb|/|'),
with an optional sign.

\begin{indpar}
\key{ratio} ::= {\em unsigned-ratio}
	    $|$ {\em sign} {\em unsigned-ratio} \\[1ex]
\key{unsigned-ratio} ::= {\em numerator} \verb|/| {\em denominator} \\[1ex]
\key{numerator} ::= {\em decimal-natural} \\[1ex]
\key{denominator} ::= {\em decimal-natural}
\end{indpar}

Ratio Examples:

\begin{indpar}[0.1in]
\tt
\begin{tabular}{l@{~~~~~}l@{~~~~~}l@{~~~~~}l@{~~~~~}l}
1/2	& -3/4		& +1,234/5	& 1,234/5,432	& -53/000 \\
\end{tabular}
\end{indpar}

If the denominator equals zero, the ratio is still a legal lexeme, but
will cause an error when it is interpreted as a number.
\verb|-53/000| is an example.

\subsubsubsection{Scientific Numbers}
\label{SCIENTIFIC-NUMBERS}

A {\em scientific-number} is a {\em decimal-number} or a {\em radix-number}
followed by an exponent.

\begin{indpar}
\key{scientific-number} \begin{tabular}[t]{rl}
                        ::= & {\em decimal-number} {\em exponent} \\
			$|$ & {\em radix-number} {\em exponent}
		        \end{tabular} \\[1ex]
\key{exponent} ::= {\em exponent-indicator}
		   {\em exponent-sign-and-digits} \\[1ex]
\key{exponent-indicator} ::= \verb|e| $|$ \verb|E| $|$ \verb|^| \\[1ex]
\key{exponent-sign-and-digits} ::= {\em decimal-digits}
			    $|$ {\em sign} {\em decimal-digits}
\end{indpar}

Scientific Number Examples:

\begin{indpar}[0.1in]
\tt
\begin{tabular}{l@{~~~~~}l@{~~~~~}l@{~~~~~}l@{~~~~~}l}
123e0		& -123e+2	& +123e-321	& 1,234e9 \\
123E0		& -123E+2	& +123E-321	& -0.123,456e-3	\\
123\verb|^|0	& -123\verb|^|+2
				& -0.123,456e-3 & -1,234.567890\verb|^|6 \\
X\#a9\#e0	& B\#1011\#e-3	& O\#0.7753\#e-5
					    & X\#0.FE8A,932B,E\#\verb|^|+5 \\
\end{tabular}
\end{indpar}

Note that exponents cannot contain commas.

\subsubsubsection{Unit Numbers}
\label{UNIT-NUMBERS}

\ikey{Unit-numbers}{unit-number} are just {\em decimal-numbers}
with a {\em unit-indicator} prefixed or postfixed.  A {\em sign} may be before
or after a prefix {\em unit-indicator}.

\begin{indpar}
\key{unit-number} \begin{tabular}[t]{rl}
		  ::= & {\em pre-unit-indicator} {\em unit-base-number} \\
		  $|$ & {\em sign} {\em pre-unit-indicator}
		  		   {\em unsigned-unit-base-number} \\
		  $|$ & {\em unit-base-number} {\em post-unit-indicator}
		  \end{tabular} \\[1ex]
\key{unit-base-number} ::= {\em decimal-number} \\[1ex]
\key{unsigned-unit-base-number} ::= {\em unsigned-decimal-number} \\[1ex]
\key{pre-unit-indicator} ::= \verb|$| $|$ {\tt \pounds} \\[1ex]
\key{post-unit-indicator} ::= \verb|'| $|$ \verb|"| $|$ \verb|%| $|$ $^\circ$
			  $|$ \verb|i| $|$ \verb|j| $|$ \verb|k|
			      \\[1ex]
\key{unit-indicator} ::= {\em pre-unit-indicator}
		     $|$ {\em post-unit-indicator}
\end{indpar}

However, unit numbers are not lexemes.

{\bf Unit Number Rule.}\index{Unit Number Rule}
If a {\em unit-number} is to be output as a {\em word} lexeme, then instead
the {\em unit-number} is split into two {\em word} lexemes, one of which
is the {\em real-number} part and the other of which is a 1-character
{\em word} consisting of the {\em unit-indicator}.

Unit Number Examples:

\begin{indpar}[0.1in]
\begin{tabular}{@{}l@{\hspace*{0.3in}}l}
Unit Number		& Lexeme Sequence	\\[1ex]
\tt \$5.71		& \tt \$~~5.71 \\
\tt \pounds -5.71	& \tt \pounds ~~-5.71 \\
\tt 15'			& \tt 15~~' \\
\tt -2.543"		& \tt -2.543~~" \\
\tt 72\%		& \tt 72~~\% \\
\tt 0.5i		& \tt 72~~i \\
\end{tabular}
\end{indpar}

The units `\ttmkey{i}{as unit}', `\ttmkey{j}{as unit}', and
`\ttmkey{k}{as unit}' are used to specify complex imaginary or quaternion
numbers: see \pagref{IMAGINARY-UNITS}.

Note that {\em radix-numbers}, {\em ratios}, and {\em scientific-numbers}
cannot have units as part of their pre-lexeme.  This is to reduce confusion:
the units can always be given separated by a space from the number,
as in `{\tt \pounds} \verb|O#7601#|', `\verb|$ 3/4|', and `\verb|5e3 i|'.

\subsubsection{Post Separators}
\label{POST-SEPARATORS}

Informally, a post-separator is a 1-character separator that immediately
follows a word and could be part of that word.  Examples are the comma, period,
and the exclamation point.  There are two kinds of post separators:
weak and strong.  The Post Separator Rule
given below tells when a {\em pre-word} ending
in a {\em post-separator-character}
must be split into a smaller {\em pre-word} and a 1-character
{\em separator}.  This rule makes reference to {\em words} that are
{\em numbers}, saying that splitting is preferred when the
{\em post-separator-character} is weak if the {\em pre-word}
resulting from the split is a {\em number}.

The syntax equations required are:

\begin{indpar}
\key{separator} ::= {\em format-separator}
		$|$ {\em post-separator} \\[1ex]
\key{post-separator} ::= {\em post-separator-character} \\[1ex]
\key{post-separator-character} \begin{tabular}[t]{rl}
                     ::= & {\em strong-post-separator-character} \\
		     $|$ & {\em weak-post-separator-character}
		     \end{tabular} \\[1ex]
\key{strong-post-separator-character} ::=     \verb|,|
			                $|$ \verb|;|
					\\[1ex]
\key{weak-post-separator-character} ::=     \verb|!|
				        $|$ \verb|?|
			                $|$ \verb|:|
			                $|$ \verb|.|
					\\[1ex]
\key{pre-word} ::= {\em word}
		$|$ {\em pre-word} {\em post-separator-character}
\end{indpar}

{\bf Post Separator Rule.}\index{Post Separator Rule}
A {\em pre-word} of 2 or more characters
that ends with a {\em post-separator-character}
is split into a smaller {\em pre-word} and a {\em post-separator}
if (1) the {\em post-separator-character} is strong, or if
(2) the smaller {\em pre-word} ends with a
{\em strong-post-separator-character}, or if
(3) the smaller {\em pre-word} is a {\em number}, or if
(4) the smaller {\em pre-word} does 
\underline{not} contain a copy of the {\em post-separator-character}.
If the {\em pre-word} is not split, it becomes a {\em word} lexeme.

The following rule handles the case of 1-character {\em pre-words}.

{\bf Isolated Post Separator Rule.}\index{Isolated Post Separator Rule}
A {\em pre-word} that consists of a single {\em post-separator-character}
is a {\em post-separator} lexeme.


Examples:

\begin{indpar}[0.1in]
\begin{tabular}{@{}l@{\hspace*{0.3in}}l@{~~~}l@{\hspace*{0.6in}}l}
Pre-Word		& \multicolumn{2}{@{}l@{}}{Lexeme Sequence}
							& Splits?	\\[1ex]
\verb|hello.|		& \tt hello 		& \tt .	& yes		\\
\verb|X,|		& \tt X 		& \tt ,	& yes		\\
\verb|h.e.l.l.o.|	& \tt h.e.l.l.o.	&	& no		\\
\verb|e.g.|		& \tt e.g.		&	& no		\\
\verb|5.0.|		& \tt 5.0		& \tt .	& yes		\\
\verb|1,234.5,|		& \tt 1,234.5		& \tt ,	& yes		\\
\verb|1,234.5,,|	& \tt 1,234.5		& \tt ,~~,
							& yes, twice	\\
\verb|1,234.5.,|	& \tt 1,234.5		& \tt .~~,
							& yes, twice	\\
\verb|1,234.5,.|	& \tt 1,234.5		& \tt ,~~.
							& yes, twice	\\
\verb|help!|		& \tt help		& \tt !	& yes		\\
\verb|!help!|		& \tt !help!		&	& no		\\
\verb|help.!|		& \tt help		& \tt .~~!
							& yes, twice	\\
\verb|.help.!|		& \tt .help.		& \tt !
							& yes, once	\\
\verb|!.help.!|		& \tt !.help.!		&	& no		\\
\end{tabular}
\end{indpar}

\subsubsection{Indentation Lexemes}
\label{INDENTATION-LEXEMES}

{\em Indentation-lexemes} are implied by indentation.  The
\mkey{indentation}{of line} of a line is the
the number of {\em space} characters in the {\em white-space}
pre-lexeme just before the first non-{\em white-space} pre-lexeme
of the line.  Note that lines cannot be empty; empty lines are
merged into {\em white-space} pre-lexemes.

At any given point in the scan converting pre-lexemes to lexemes, there is a
stack of indentation records, called the \key{indentation stack}.
An \key{indentation record} is a number of
columns, a pre-lexical context, and a flag.
The number of columns is called the \key{indentation}.
The flag is called the \key{implicit bracket flag}, and indicates whether
or not an implied \verb|{| lexeme was inserted at the same time the
indentation record was pushed onto the indentation stack.

Initially the indentation stack contains a single indentation record with
0 indentation, the outermost pre-lexical context,
and an off implicit bracket flag.
The stack cannot become empty; any operation that would
pop the last indentation off the stack announces an error and leaves the
stack alone.  Thus the bottommost indentation on the stack is always
the same.

The indentation in the indentation record at the top of the indentation
stack is called the \key{current indentation}, and the pre-lexical
context in that record is called the \key{current indentation context}.

The rules for creating indentation records are such that the pre-lexical
contexts in such records are either pre-lexical contexts of some \verb|{ }|
bracket pair, or are the outermost pre-lexical context, which therefore behaves
like the context of a \verb|{ }| bracket pair as far as indentation is
concerned.  A line beginning is said to be in a particular pre-lexical
context if the first lexeme on the line, were it not a {\em closing-mark},
would be in that pre-lexical context.

{\bf Semi-Colon Rule.}\index{Semi-Colon Rule}
If the first non-{\em white-space} pre-lexeme of a line is not \verb|}|,
if the beginning of the line is in the current indentation context,
if the indentation of the line is the current indentation,
and if the last lexeme output was not
a \verb|;| {\em separator} lexeme or a \verb|{| {\em opening-mark}
lexeme (including an implied {\em opening-mark} as in the
Implicit-Bracket Rule below), then a \verb|;| implied {\em separator}
lexeme is output
before any lexemes generated by the line are output.

{\bf Explicit-Bracket Rule.}\index{Explicit-Bracket Rule}
If the last pre-lexeme of a line is \verb|{|, then an indentation record
is pushed onto the indentation stack just after the \verb|{| lexeme
is output.  The indentation in the record is the indentation of the next
line, the pre-lexical context is that in effect just after the \verb|{|,
and the implicit bracket flag is off.
The pushed indentation record is popped just after its pre-lexical context
ends; that is, just before reading the \verb|}| pre-lexeme that matches the
\verb|{| pre-lexeme that pushed the indentation record.

{\bf Implicit-Bracket Rule.}\index{Implicit-Bracket Rule}
If the last lexeme output for a line would be a separator \verb|:| in
the current indentation context,
then a \verb|{| implied {\em opening-mark} lexeme
is output instead of the \verb|:|, and an indentation
record is pushed onto the indentation stack.  The indentation of the
record is the indentation of the next line, the implicit bracket flag
of the record is on, and the
pre-lexical context of the record is that of the \verb|:|.
The pushed record is popped (1) just before the first line such that
the line beginning in the record's pre-lexical context
and the line's indentation is less than the record's indentation,
or (2) at the end of the input pre-lexeme stream.
When the record is popped, a \verb|}| implied {\em closing-mark}
lexeme is output.  It is an error if this implied {\em closing-mark}
does not match the implied {\em opening-mark} associated with the popped
record.

In order to avoid subtle errors created by indentation, there
is a {\tt minimum-indentation} parameter and the following rule.

{\bf Minimum-Indentation Rule.}\index{Minimum-Indentation Rule}
The indentation of any line whose beginning is in the current indentation
context (the first pre-lexeme of the line would be in the current indentation
context if it did not end a sub-context) must equal the current indentation
or differ from it by at least the value of the
\ttkey{minimum-indentation} parameter.  The {\tt minimum-indentation}
parameter defaults to {\tt 4}.


Indentation Lexeme Examples:

\begin{indpar}[0.1in]
\tt \ttbrackets
\begin{tabular}{@{}p{2in}@{~~~~~~~~~~}l}
{\rm Input String}		& {\rm Output Lexemes} \\[1ex]
hi \{ x; y z; w \}		& \tt hi \{ x ; y z ; w \} \\[1ex]
hi \{				& hi \{ \\
~~~~x				& x \\
~~~~y z				& ; y z \\
~~~~w \}			& ; w \}
\end{tabular}
\end{indpar}
\begin{indpar}[0.1in]
\tt \ttbrackets
\begin{tabular}{@{}p{2in}@{~~~~~~~~~~}l}
hi				& hi \\
\{				& \{ \\
~~~~x				& x \\
~~~~y z				& ; y z \\
~~~~w				& ; w \\
\}				& \}
\end{tabular}
\end{indpar}
\begin{indpar}[0.1in]
\tt \ttbrackets
\begin{tabular}{@{}p{2in}@{~~~~~~~~~~}l}
hi:				& hi \{ \\
~~~~x				& x \\
~~~~y z				& ; y z \\
~~~~w				& ; w \\
				& \}
\end{tabular}
\end{indpar}
\begin{indpar}[0.1in]
\tt \ttbrackets
\begin{tabular}{@{}p{2in}@{~~~~~~~~~~}l}
hi \{				& hi \{ \\
~~~~x				& x \\
~~~~y `this			& ; y ` this \\
is another pre-lexical		& is another pre-lexical \\
context.			& context . \\
' foo bar			& ' foo bar \\
~~~~w \}			& ; w \}
\end{tabular}
\end{indpar}
\begin{indpar}[0.1in]
\tt \ttbrackets
\begin{tabular}{@{}p{2in}@{~~~~~~~~~~}l}
hi \{				& hi \{ \\
~~~~x				& x \\
~~~~\{				& ; \{ \\
ho				& ho \\
hum \} y			& ; hum \} y \\
~~~~z \}			& ; z \}
\end{tabular}
\end{indpar}


\subsection{Lexical Parsing}
\label{LEXICAL-PARSING}

After pre-lexemes are scanned to produce lexemes, the lexemes are
scanned to perform the following:

\begin{indpar}[1in]
Lexeme Replacement (Spelling Regularization, Possessive Splitting) \\
Number Unit Grouping
\end{indpar}

We describe these below in order.

The scan that performs the above actions is called
\key{lexical parsing}\index{parsing!lexical},
because it does initial parse steps
at the level of individual lexemes and short sequences of lexemes, and
because it is controlled by definitions in the parsing stack
(\pagref{PARSING-STACK}).

\subsubsection{Lexeme Replacement}
\label{LEXEME-REPLACEMENT}

The lexeme replacement process replaces one lexeme by a string of zero or
more lexemes.  Replacement may be done by either a dictionary or a function.
Replacement is controlled by \key{lexeme replacement definition}s
in the parsing stack (\pagref{PARSING-STACK}).

There are two kinds of lexeme replacement definitions:
dictionary and function.  These have the syntax:

\begin{indpar}
\ikey{LEXEME-DEFINITION}{lexeme-definition}
		     \begin{tabular}[t]{rl}
                     ::= & {\em LEXEME-DICTIONARY-DEFINITION} \\
		     $|$ & {\em LEXEME-FUNCTION-DEFINITION}
		     \end{tabular} \\[1ex]
\ikey{LEXEME-DICTIONARY-DEFINITION}{lexeme-dictionary-definition} \\
	\hspace*{0.5in}::= \begin{tabular}[t]{@{}l@{}l@{}}
	                   \verb|define | & \verb|lexeme dictionary|
                                            {\em DICTIONARY-NAME} \\
				& {\em LEXEME-DICTIONARY-ENTRIES}
		     \end{tabular} \\[1ex]
\ikey{DICTIONARY-NAME}{dictionary-name} ::=
	\verb|[|
	\{ {\em word} $|$ {\em separator} \}
	\{ {\em word} $|$ {\em separator} \}$^\star$
	\verb|]| \\[1ex]
\ikey{LEXEME-DICTIONARY-ENTRIES}{lexeme-dictionary-entries}
                    ::= \verb|(| {\em lexeme-dictionary-entry-list} \verb|)|
		    \\[1ex]
\key{lexeme-dictionary-entry-list} \\
		     \hspace*{0.5in}\begin{tabular}[t]{rl}
                     ::= & {\em LEXEME-DICTIONARY-ENTRY} \\
		     $|$ & {\em lexeme-dictionary-entry-list} \verb|,|
		           {\em LEXEME-DICTIONARY-ENTRY}
		     \end{tabular} \\[1ex]
\ikey{LEXEME-DICTIONARY-ENTRY}{lexeme-dictionary-entry} ::=
	\verb|[| {\em replaced-lexeme} \verb|=>| {\em replacing-lexemes}
	\verb|]| \\[1ex]
\key{replaced-lexeme} ::=
	\{ {\em word} $|$ {\em separator} \} \\[1ex]
\key{replacing-lexemes} ::=
	\{ {\em word} $|$ {\em separator} \}
	\{ {\em word} $|$ {\em separator} \}$^\star$ \\[1ex]
\ikey{LEXEME-FUNCTION-DEFINITION}{lexeme-function-definition} \\
	\hspace*{0.5in}::= \verb|define lexeme function|
				{\em FUNCTION-NAME} \\[1ex]
\ikey{FUNCTION-NAME}{function-name} ::=
	\verb|[|
	\{ {\em word} $|$ {\em separator} \}
	\{ {\em word} $|$ {\em separator} \}$^\star$
	\verb|]|
\end{indpar}

which correspond to the expression definitions
(\secref{EXPRESSION-DEFINITIONS}):

\begin{indpar}
\begin{verbatim}
define lexeme dictionary NAME ENTRIES <-- n-tuple ENTRIES

define lexeme function NAME <--
\end{verbatim}
\end{indpar}

A \key{lexeme dictionary entry} gives a lexeme that is to be replaced
and a sequence of lexemes that replace it.  For example, the entry

\begin{center}
\verb|[ people ==> person ]|
\end{center}

causes the word `\verb|people|' to be replaced by the word `\verb|person|'.

A \key{lexeme function} is a function that is called with a lexeme as its
single argument and which returns either `\verb|false|' if the lexeme is
not to be replaced or returns an expression consisting of a sequence of
lexemes that are to replace the lexeme otherwise.  For example, given the
lexeme function definition

\begin{indpar}
\begin{verbatim}
define lexeme function [replace people]
\end{verbatim}
\end{indpar}

and the expression definition

\begin{indpar}
\begin{verbatim}
replace people X <-- {
    if X == people:
        value = [person]
    else:
        value = false }
\end{verbatim}
\end{indpar}

an appearance of the lexeme `\verb|fie|' will evaluate the
expression `\verb|replace person fie|' which will return `\verb|false|'
to avoid replacing the lexeme `\verb|fie|'.

It is possible to temporarily void lexeme replacement definitions by
placing canceling undefinitions in the parsing stack.  These have the
syntax:

\begin{indpar}
\ikey{LEXEME-UNDEFINITION}{lexeme-undefinition}
		     \begin{tabular}[t]{rl}
                     ::= & {\em LEXEME-DICTIONARY-UNDEFINITION} \\
		     $|$ & {\em LEXEME-FUNCTION-UNDEFINITION}
		     \end{tabular} \\[1ex]
\ikey{LEXEME-DICTIONARY-UNDEFINITION}{lexeme-dictionary-undefinition} \\
	\hspace*{0.5in}::= \verb|undefine lexeme dictionary|
                                            {\em DICTIONARY-NAME} \\[1ex]
\ikey{LEXEME-FUNCTION-UNDEFINITION}{lexeme-function-undefinition} \\
	\hspace*{0.5in}::= \verb|undefine lexeme function|
				{\em FUNCTION-NAME}
\end{indpar}

which correspond to the expression definitions
(\secref{EXPRESSION-DEFINITIONS}):

\begin{indpar}
\begin{verbatim}
undefine lexeme dictionary NAME <--

undefine lexeme function NAME <--
\end{verbatim}
\end{indpar}

The following lexeme dictionaries and functions are
defined in the initial parsing stack.  Their definitions are ordered in the
stack so the first given below is at the top of the stack and is the
first that replaces lexemes.

\begin{list}{}{}

\item
\verb|define lexeme dictionary [english lexeme dictionary]|~~
This dictionary translates common English irregular plurals,
such to their singular form, and decomposes
irregular possessives to their singular decomposed form.
This dictionary also protects irregular singular forms that might be
mistaken for regular plural forms (e.g., `\verb|news|').
Some example translations are:

\begin{center}
\begin{tabular}{l}
\verb|people| $\longrightarrow$ \verb|[person]| \\
\verb|women| $\longrightarrow$ \verb|[woman]| \\
\verb|geese| $\longrightarrow$ \verb|[goose]| \\
\verb|fungi| $\longrightarrow$ \verb|[fungus]| \\
\verb|news| $\longrightarrow$ \verb|[news]| \\
\end{tabular}
\end{center}

\item
\verb|define lexeme function [english lexeme function]|~~
This function translates common English standard plurals,
such to their singular form, and decomposes
standard possessives to their singular decomposed form.
Some example translations are:

\begin{center}
\begin{tabular}{l}
\verb|boys| $\longrightarrow$ \verb|[boy]| \\
\verb|boy's| $\longrightarrow$ \verb|[boy 's]| \\
\verb|boys'| $\longrightarrow$ \verb|[boy 's]| \\
\verb|boxes| $\longrightarrow$ \verb|[box]| \\
\end{tabular}
\end{center}

\end{list}

Lexeme replacement is \underline{not} recursive: the replacement lexemes are
not themselves subject to replacement.

Because PCASL insists on mapping different forms of a word to a single
word, some subtleties of language are lost.  For example, `\verb|people|'
can be a singular word referring to a group of people, but PCASL will
standardly confuse it with `\verb|person|'.

\subsubsection{Number Unit Grouping}
\label{NUMBER-UNIT-GROUPING}

\section{Expression Parsing}\index{parsing!expression}
\label{EXPRESSION-PARSING}

An expression is a sequence of words, separators, and subexpressions.
A subexpression is a pair of matched lexemes and all the lexemes in
between.

Expressions and subexpressions can contain operators.  When they do,
matched implied parentheses are inserted into the expressions or subexpressions
according to rules of operator precedence and associativity, and these
implied parentheses create new subexpressions.

Expressions and subexpressions that do not contain operators may contain
argument lists and qualifying phrases.  The order in which qualifying
phrases appear does not matter, and sometimes the order of arguments
in an argument list does not matter.

Expressions containing operators are restructured, both by inserting
implied parentheses and in other ways.  Subexpressions surrounded by
particular matching lexemes, such as \verb|`'|, are restructured in
special ways.

The process of restructuring expressions is called
\key{expression parsing}\index{parsing!expression}.
Expression parsing takes an expression as input, and produces as output a raw
expression that contains nothing that will trigger further restructuring.

\subsection{Subexpressions}
\label{SUBEXPRESSIONS}

The first step in parsing is to identify \skey{subexpression}s within
an expression, without restructuring the expression, by applying
the following syntax equations.

\begin{indpar}
\key{expression} ::= {\em expression-item}$^\star$ \\[1ex]
\key{expression-item} ::= {\em word} $|$ {\em separator}
				     $|$ {\em subexpression} \\[1ex]
\key{subexpression} ::= {\em opening-mark} {\em expression}
			{\em closing-mark}
\end{indpar}

{\bf Subexpression Rule.}\index{Subexpression Rule}
The {\em opening-mark} lexeme that begins a subexpression must lexically match
(\secref{LEXICAL-MATCHING}) the {\em closing-mark} lexeme that
ends a subexpression.

\subsection{Expression Structure Overview}
\label{EXPRESSION-STRUCTURE-OVERVIEW}

Expression structure is affected by five special marks.
The \key{optional argument mark} `\ttkey{?:?}' is used in expression
definitions
to separate arguments that are required from those that can be optional,
as in the expression `\verb|find-max LIST ?:? COMPARATOR|', in which the
the `\verb|LIST|' argument is required but the `\verb|COMPARATOR|' argument
is optional.
The \key{reorder mark} `\ttkey{<:>}' is used in expression definitions
to separate arguments whose order can be switched,
as in the expression `\verb|element LIST <:> INDEX|' in which the
`\verb|LIST|' and `\verb|INDEX|' argument can be switched.
The \key{remainder mark} `\ttkey{::>}' precedes a final argument in an
expression definition that represents the list of remaining arguments in a use
of the definition, as in the expression `\verb|max ::> ARGS|',
where \verb|ARGS| represents
the list of all the arguments given to \verb|max|.
The \key{required qualifier mark}\index{qualifier mark!required}
`\ttkey{@@}' signals that the following
expression item is a required qualifier, as in the expression
`\verb|sort x @@ with order ascending |',
and is implied by qualifiers (e.g. \verb|with|, \verb|has|)
in an expression.
The \key{optional qualifier mark}\index{qualifier mark!optional}
`\ttkey{??}' signals that the following
expression item is an optional qualifier, as in the expression
`\verb|sort X ?? with order O|'.  When default values are provided in
definitions, required qualifier marks may be converted automatically to
optional qualifier marks and an optional argument mark may be inserted into an
argument list automatically (\secref{TBD}).

Expressions are restructured if they contain operators (e.g.,
\verb|+|,
\verb|-|,
\verb|*|,
\verb|/|),
qualifiers (e.g.,
\verb|with|,
\verb|has|),
or qualifier shortcuts (e.g., \verb|ascending|, which is a shortcut for
`{\tt with order ascending}').
A \key{raw expression} (\secref{RAW-EXPRESSIONS})
is a particular kind of expression that will not be
restructured.  The only special marks that may be in a raw expression
are the five special marks:
the optional argument mark (\verb|?:?|), the reorder mark (\verb|<:>|),
the remainder mark (\verb|::>|),
and the qualifier marks (\verb|@@|, \verb|??|).
Translating
an expression into a raw expression is called \key{parsing}.

Some operators are matchfix operators, which surround their single
argument like parentheses.  Some matchfix operators have special
affects on parsing.  An informal summary of the standard matchfix
operators is:

\begin{center}
\begin{tabular}{l@{~~~~~~}l@{~~~~~~}l@{~~~~~~}l@{~~~~~~}l}
		& \multicolumn{4}{c}{Turns} \\ \cline{2-5}
		& operators	& \verb/|/	& \verb|<:> ::>| \\
Brackets	& qualifiers	& text		& \verb|@@  ??|
								& eval \\[1ex]

\verb|(|\ldots\verb|)|
		& --		& --		& --		& -- \\
\verb|{|\ldots\verb|}|
		& on		& off		& on		& off \\
\verb|[|\ldots\verb|]|
		& off		& off		& on		& off \\
\verb|{{|\ldots\verb|}}|
		& on		& off		& on		& -- \\
\verb|[[|\ldots\verb|]]|
		& off		& off		& on		& -- \\
\verb|<<|\ldots\verb|>>|
		& on		& off		& on		& on \\
\verb|<<<|\ldots\verb|>>>|
		& on		& off		& on		& on twice \\
\verb|<<<<|\ldots\verb|>>>>|
		& on		& off		& on		& on thrice \\
\verb|`|\ldots\verb|'|
		& off		& on		& off		& off \\
\verb|``|\ldots\verb|''|
		& off		& on		& off		& off \\
\verb|```|\ldots\verb|'''|
		& off		& on		& off		& off \\
\end{tabular}
\end{center}

Other matching lexemes may be made to affect parsing by introducing
operator definitions (\secref{OPERATOR-DEFINITIONS}).

Parsing is controlled by the \key{parsing stack},\label{PARSING-STACK}
which contains definitions of parsers, operators, qualifiers, and
qualifier shortcuts, as well as definitions related to
lexical parsing (\pagref{LEXICAL-PARSING}).
An operator may push definitions onto the parsing
stack.  After the expression containing the operator has been parsed,
these definitions will be popped from the parsing stack.

The following is a list of the kinds of definitions that may be in
the parsing stack:

\begin{center}
\begin{tabular}{ll}
parser & \pagref{PARSER-DEFINITION} \\
operator & \secref{OPERATOR-DEFINITIONS} \\
qualifier & \ref{TBD} \\
qualifier shortcut & \ref{TBD} \\
lexeme & \secref{LEXEME-REPLACEMENT}
\end{tabular}
\end{center}

A \key{parser definition}\label{PARSER-DEFINITION}
just names a parser.  The most recent parser
definition in the parsing stack names the parser used to parse expressions.
There are three standard \ikey{parsers}{parser!standard}:
the \key{marks parser} \ttkey{-MARKS-PARSER-},
where only the optional argument mark (\verb|?:?|), reorder mark (\verb|<:>|),
remainder mark (\verb|::>|), required qualifier mark (\verb|@@|),
and optional qualifier mark (\verb|??|), are recognized;
the \key{operators parser} \ttkey{-OPERATORS-PARSER-},
where these marks
are recognized along with operators, qualifiers, and qualifier shortcuts;
and the \key{text parser} \ttkey{-TEXT-PARSER-},
where only the format-separator (\verb/|/) and punctuation are recognized.

From the parsing point of view there are several types of
expression restructuring that can occur.  One, text restructuring,
is done by the text parser (\secref{TEXT-PARSING}), which is invoked by
\verb|`'| quotes, and does things like turn
\verb|`I am!'| into \verb|[sentence I am !]|.  Other kinds of
restructuring are done by the operators parser (\secref{OPERATORS}).
The operators `\verb|+|', `\verb|-|', and `\verb|*|' in the expression
\verb|x + y - 5 * z| cause this expression to be restructured as
\verb|[- [+ x y] [* 5 z]]|.  The qualifier `\verb|with|' in the
expression \verb|sort x with order ascending| causes this expression
to be restructured as \verb|[sort x @@ with order ascending]|.
And the qualifier shortcut `\verb|ascending|' in the expression
\verb|sort x| \verb|ascending| causes this expression to be
restructured as \verb|[sort x @@ with order ascending]|.

In expressions that cannot contain operators,
words and separators that would be recognized elsewhere as operators,
qualifiers, or qualifier shortcuts are not recognized as such.  Thus in
\verb|[word +]| the \verb|+| is not recognized as an operator, and
in \verb|[item ascending]| \verb|ascending| is not recognized as
a qualifier shortcut.

Some of the matchfix operators above affect evaluation (eval).
Evaluation differs from parsing in that it is managed by a
counter and not a stack; if evaluation is
turned off twice, it must be turned on twice to be in effect.
Thus in a context in which evaluation is on, the \verb|X| in
\verb|{ <<X>> = 5 }| and \verb|{ y = `<< <<X>> >>' }| will be evaluated
but the \verb|X| in \verb|{ y = `<<X>>' }| will not be.  There is
an \key{evaluation counter} value that is
incremented (decremented) by an opening mark
that turns evaluation on (off), and decremented (incremented) by the
corresponding closing mark.  This counter sticks at +1: matching
lexemes that would increment the counter have no affect if the counter
is at +1.

The brackets `\verb|<<<|\ldots\verb|>>>|' are equivalent to
`\verb|<< <<|\ldots\verb|>> >>|' and thus turn evaluation on twice.
Thus \verb|{ y = `<< <<X>> >>' }| is equivalent to \verb|{ y = `<<<X>>>' }|.
Similarly the brackets `\verb|<<<<|\ldots\verb|>>>>|'
are equivalent to `\verb|<< << <<|\ldots\verb|>> >> >>|' and
turn evaluation on thrice.

Evaluation is not part of parsing; the evaluation counter is
used during expression evaluation, which occurs after expression parsing.
Parsing does insert the wrappers (\secref{OPERATOR-WRAPPER})
\ttkey{-EVAL-ON-} and \ttkey{-EVAL-OFF-} around expressions to inform
the evaluator when to increment or decrement the evaluation counter.

\subsection{Raw Expressions}
\label{RAW-EXPRESSIONS}

A raw expression is an expression that contains no operators,
qualifiers, qualifier shortcuts, or text parsing contexts that
induce restructuring.
Raw expressions can be directly represented using the \verb|[]|
matching lexemes and the \verb|?:?| optional argument,
\verb|<:>| reorder, \verb|::>| remainder,
\verb|@@| required qualifier, and \verb|??| optional qualifier marks.

When an expression is parsed, the output is a raw expression.
The following are examples, all of which are assumed to
appear in the context of \verb|{}| matching lexemes (so
operators, qualifiers, and qualifier shortcuts
are turned on and text parsing is turned off):

\begin{center}
\begin{tabular}{ll}
Input Expression	& Raw Expression Equivalent \\[1ex]
\verb|3 * x * x + 5|	& \verb|[+ [* [* 3 x] x] 5]| \\
\verb|x = y = 10|	& \verb|[= x [= y 10]]| \\
\verb|0 < x <= 5 |	& \verb|[-COMPARE- 0 < x <= 5]| \\
\verb|x = `Hello <<y>>.' |
			& \begin{tabular}[t]{@{}l}
	    \verb|[= x [-EVAL-OFF-| \\
	    \verb|      [sentence Hello| \\
	    \verb|                [-EVAL-ON- y]| \\
	    \verb|                .]]]| \\
	    \end{tabular}
\end{tabular}
\end{center}

The syntax of raw expressions is defined as follows:

\begin{indpar}
\key{raw-expression} ::= {\em word} $|$ {\em separator}
				     $|$ {\em raw-subexpression} \\[1ex]
\key{raw-subexpression} ::=
	\begin{tabular}[t]{@{}ll}
	\verb|[| & {\em raw-expression-head} \\
		 & {\em raw-argument-list} \\
		 & {\em raw-qualifier-phrase}$^\star$ \verb|]| \\
	\end{tabular} \\[1ex]
\key{raw-expression-head} ::= {\em raw-expression} \\[1ex]
\key{raw-argument-list}
        \begin{tabular}[t]{@{}rl}
	::= & {\em raw-base-argument-list}
	      {\em raw-argument-remainder-option} \\
	$|$ & \begin{tabular}[t]{@{}l}
	      {\em raw-base-argument-list}
	      {\em optional-argument-mark} \\
	      {\em non-empty-raw-base-argument-list} \\
	      {\em raw-argument-remainder-option}
	      \end{tabular}
	\end{tabular} \\[1ex]
\key{optional-argument-mark} ::= \verb|?:?| \\[1ex]
\key{raw-base-argument-list}
	::= {\em empty} $|$ {\em non-empty-raw-base-argument-list} \\[1ex]
\key{raw-non-empty-base-argument-list}
        \begin{tabular}[t]{@{}rl}
	::= & {\em raw-argument} \\
	    & \{ {\em reorder-mark-option} {\em raw-argument} \}$^\star$
	\end{tabular} \\[1ex]
\key{reorder-mark-option} ::= {\em empty} $|$ \verb|<:>| \\[1ex]
\key{raw-argument-remainder-option} ::=
	{\em empty} $|$ \verb|::>| {\em raw-argument} \\[1ex]
\key{raw-argument} ::= {\em raw-expression} \\[1ex]
\key{raw-qualifier-phrase} ::=
	{\em qualifier-mark} {\em raw-qualifier-head}
			     {\em raw-argument-list} \\[1ex]
\key{qualifier-mark} ::= {\em required-qualifier-mark}
			 $|$ {\em optional-qualifier-mark} \\[1ex]
\key{required-qualifier-mark} ::= \ttkey{@@} \\[1ex]
\key{optional-qualifier-mark} ::= \ttkey{??} \\[1ex]
\key{raw-qualifier-head} ::= {\em raw-expression}
\end{indpar}

The order of {\em raw-qualifier-phrases}
in a {\em raw-expression} does not matter.
The order of arguments separated by {\em reorder-marks} (\verb|<:>|) does
not matter.  The {\em optional-argument-mark} (\verb|?:?|),
{\em reorder-mark} (\verb|<:>|), {\em remain\-der-mark}
(\verb|::>|), and {\em optional-qualifier-mark} (\verb|??|)
can only occur in a pattern (\secref{CALL-UNIFICATION}).

\subsection{Operators}
\label{OPERATORS}

Operators restructure expressions in which they occur.  Operators
are defined by operator definitions that can be added to the
parsing stack by some operators.

\subsubsection{Operator Definitions}
\label{OPERATOR-DEFINITIONS}

\ikey{Operator definitions}{operator definition} can be pushed onto
the parsing stack, and are used by the operators parser.
An operator definition specifies for each operator the following:

\begin{indpar}[1in]
Fixity \\
Name \\
Precedence \\
Associativity \\
Parser \\
Control Flags \\
Wrapper \\
Subdefinitions
\end{indpar}


\subsubsubsection{Operator Fixity}\ttmindex{fixity}{of operator}
\label{OPERATOR-FIXITY}
An operator has one of the following fixities:

\begin{indpar}
\begin{tabular}{@{}p{1in}@{~~~}p{4in}@{}}
\ttkey{infix}		& E.g., \verb|+| in \verb|x + 5|.\\
\ttkey{prefix}		& E.g., \verb|-| in \verb|- 5|.\\
\ttkey{postfix}		& E.g., \verb|!| in \verb|x!|.\\
\ttkey{matchfix}	& E.g., \verb/[| |]/ in \verb/[| x - 5 |]/.
\end{tabular}
\end{indpar}

An infix operator is placed between its two operands;
a prefix operator is placed before its one operand;
a postfix operator is placed after its one operand;
and a matchfix operator surrounds its one operand.
Operands of prefix, infix, and postfix
operators may not be empty; but the operand of a matchfix operator
may be empty.
An infix operator may not begin or end an expression;
a prefix operator must begin an expression;
a postfix operator must end an expression;
and a matchfix operator has two parts: an opening operator
that must begin an expression and a closing operator that must end
the same expression.

\subsubsubsection{Operator Name}\ttmindex{name}{of operator}
\label{OPERATOR-NAME}
An operator definition has a sequence of lexemes that is the
{\em name} of the operator.  The operator
inside a subexpression is just this sequence of lexemes, except for
matchfix operators, which have two sequences of lexemes, an
\key{opening operator name} that must begin the subexpression and
a \key{closing operator name} that must end the subexpression.

By abuse of language, the term `\key{operator}' is often used
as a synonym for `operator name'.  Similarly `\key{opening operator}'
is used as a synonym for `opening operator name' and `\key{closing operator}'
is used as a synonym for `closing operator name'.

The opening and closing names of a matchfix
operator are bundled into a single operator name that is
sequence of lexemes consisting of
the opening operator name followed by a `\verb/.../' lexeme
followed by the closing operator name.
For example, the
matchfix operator named `\verb/[ | ... | ]/' permits subexpressions
like `\verb/[| x - 5 |]/'.  Here the opening operator name
is `\verb/[|/' and the closing operator name is `\verb/|]/'.

Opening and closing marks cannot be part of an operator name
unless the operator is a matchfix operator.
If opening and closing marks are part of a matchfix operator
name, they must begin and end the name, and they must \underline{not} be
parentheses \verb|()|.

When a subexpression is tested
for matchfix operators, any opening and closing marks that begin and
end the subexpression are included in the subexpression, unless they are
parentheses \verb|()|, which are always removed (recursively) when they
begin and end a subexpression, and are thus invisible.  Thus
the matchfix operator `\verb/| ... |/' could be invoked by
the expression `\verb/x + (| y |)/', but not by the expression
`\verb/x + [| y |]/'.  Operator names are not permitted
to include parentheses \verb|()|, so names such as `\verb/(| ... |)/'
are erroneous.
Generally, including extra pairs of parentheses
\verb|()| has no affect on expression parsing or expression meaning,
though the explicit parentheses in `\verb/[(|x|)]/' will prevent 
a matchfix operator named `\verb/[| ... |]/' from being recognized.

\subsubsubsection{Operator Precedence}\ttmindex{precedence}{of operator}
\label{OPERATOR-PRECEDENCE}
The precedence of an operator is an integer.  Precedence is used in
selecting which infix operators in an expression to consider
(\secref{OPERATOR-SELECTION}).
Only infix operators have precedence.

\subsubsubsection{Operator Associativity}\ttmindex{associativity}{of operator}
\label{OPERATOR-ASSOCIATIVITY}
An infix operator has an {\em associativity} that is
a sequence of lexemes.
Three associativities, {\tt left}, {\tt right}, and
{\tt none}, have special meaning.  If an infix operator has some
other associativity, it is said to have \key{named associativity}.
Only infix operators have associativity.

If more than one infix operator of lowest precedence
is selected in an expression by the operator
selection algorithm\EOL (\secref{OPERATOR-SELECTION}),
all the infix operators that are selected
must have the identical operator definitions except for operator names,
and the associativity in these definitions
must not be {\tt none}\ttmindex{none}{associativity}.

If all operators are {\tt left} associative%
\index{left associative@{\tt left} associative},
all the infix operators but the rightmost are deselected,
so it is as if operators to the left had
higher precedence than those to the right.

If all operators are {\tt right} associative%
\index{right associative@{\tt right} associative},
all the infix operators but the leftmost are deselected,
so it is as if operators to the right had
higher precedence than those to the left.

If all operators are of \key{named associativity}, all remain selected, and 
implicitly parenthesized subexpressions will be created between the
operators as well as between the beginning of the expression and the
first operator and between the last operator and the end of the expression.
Then the associativity, as a lexeme sequence, is prepended to the expression.
Thus if the infix operators \verb|<| and \verb|<=| have associativity
\verb|-COMPARE-|, the expression \verb|0 < x - 3 <= 5| will
be rewritten as \verb|[-COMPARE- 0 < (x - 3) <= 5]|.

\subsubsubsection{Operator Parser}\ttmindex{parser}{of operator}
\label{OPERATOR-PARSER}

A parser is the name of the function that is called with an expression
as its single argument in order to parse the expression.

Parsers can be pushed onto the parsing stack.
The parser used for subexpressions of an expression is the parser
nearest the top of the parsing stack.

The \verb|parser| of an operator definition is an optional word that
names a parser.  If present, the word is pushed onto the parsing
stack before subexpressions are parsed, and popped from the parsing stack
after subexpressions have been parsed.

The following are standard parsers:

\begin{indpar}

The \key{operators parser} \ttkey{-OPERATORS-PARSER-}.
Operators, qualifiers, qualifier shortcuts,
and the \verb|?:?| optional argument,
\verb|<:>| reorder, \verb|::>| remainder,
\verb|@@| required qualifier, and
\verb|??| optional qualifier marks are recognized.  The special constructs
of text parsing, e.g., the \verb/|/ format separator,
are \underline{not} recognized.

The \key{marks parser} \ttkey{-MARKS-PARSER-}.
The \verb|?:?| optional argument,
\verb|<:>| reorder, \verb|::>| remainder,
\verb|@@| required qualifier, and
\verb|??| optional qualifier marks are recognized.  Operators, qualifiers,
qualifier shortcuts, and the special constructs
of text parsing, e.g., the \verb/|/ format separator,
are \underline{not} recognized.

The \key{text parser} \ttkey{-TEXT-PARSER-}.
The special constructs of text parsing (\secref{TEXT-PARSING}) are
recognized.  Operators, qualifiers, qualifier shortcuts, and
the \verb|?:?| optional argument,
\verb|<:>| reorder, \verb|::>| remainder,
\verb|@@| required qualifier, and
\verb|??| optional qualifier marks are \underline{not} recognized.

\end{indpar}

\subsubsubsection{Operator Control Flags}\ttmindex{control flag}{of operator}
\label{OPERATOR-CONTROL-FLAGS}
Operators can be associated with \skey{operator control flag}s that
affect parsing of subexpressions of the operator.

\begin{indpar}

{\bf definitional}\index{definitional flag@{\tt definitional} flag}~~~~~
This flag may only be given for a right associative infix operator.
It causes the
left operand of the operator to be inspected to see if it is a definition
that should be pushed onto the parsing stack.  This can be an operator,
qualifier, qualifier shortcut, or parser definition, or
a lexical parsing definition (\secref{TBD}).
If no, nothing special is done.
If yes, the expression containing
the infix operator is replaced by just its right operand, the definition
in the left operand
is pushed onto the parsing stack, the right operand is
parsed, and then the definition is popped off the parsing stack.

\end{indpar}

\subsubsubsection{Operator Wrapper}\ttmindex{wrapper}{of operator}
\label{OPERATOR-WRAPPER}

An operator wrapper is a raw expression containing the lexeme `\verb|...|'.
If operator selection for an expression selects a single operator with a
wrapper, the expression is restructured to be a copy of the wrapper
with its `\verb|...|' lexeme replaced by the parse of the expression
being restructured.

For example, if the matchfix operator `\verb|<< ... >>|' has the
wrapper `\verb|[-EVAL-ON- ...]|', then the expression `\verb|<<x>>|'
will be restructured to be `\verb|[-EVAL-ON- x]|'.

\subsubsubsection{Operator Subdefinitions}\ttmindex{definition}{of operator}
\label{OPERATOR-SUBDEFINITIONS}

An operator definition can contain a list of definitions
that are pushed onto the parsing stack before the subexpressions
of the operator are parsed, and are popped from the stack after the
subexpressions are parsed.  To avoid name confusion, these definitions
are called \skey{subdefinition}s.

Note that subdefinitions can include any definitions that can be 
pushed onto the parsing stack, and can also include undefinitions, that
inactivate all definitions with particular names in the stack
(\secref{TBD}).


\subsubsection{Operator Definition Syntax}
\label{OPERATOR-DEFINITION-SYNTAX}

The syntax of operator definitions is:

\begin{indpar}
\ikey{OPERATOR-DEFINITION}{operator-definition} \\
	\hspace*{0.5in}::= \ttkey{define operator}
	    \begin{tabular}[t]{l}
	    {\em OPERATOR-FIXITY} ~ {\em OPERATOR-NAME} \\
	    {\em operator-definition-option}$^\star$
	    \end{tabular} \\[1ex]
\ikey{OPERATOR-FIXITY}{operator-fixity} ::=
	\ttkey{prefix} $|$ \ttkey{infix} $|$ \ttkey{postfix}
	$|$ \ttkey{matchfix} \\[1ex]
\ikey{OPERATOR-NAME}{operator-name} ::=
	\verb|[|
	\{ {\em word} $|$ {\em separator} \}
	\{ {\em word} $|$ {\em separator} \}$^\star$
	\verb|]| \\[1ex]
\key{operator-definition-option} \begin{tabular}[t]{rl}
                     ::= &  {\em operator-precedence-option} \\
                     $|$ &  {\em operator-associativity-option} \\
                     $|$ &  {\em operator-parser-option} \\
                     $|$ &  {\em operator-flags-option} \\
                     $|$ &  {\em operator-wrapper-option} \\
                     $|$ &  {\em operator-subdefinition-option} \\
		     \end{tabular} \\[1ex]
\key{operator-precedence-option} ::=
	\ttkey{with precedence} {\em INTEGER} \\[1ex]
\key{operator-associativity-option} \\
	\hspace*{0.5in}\begin{tabular}[t]{rl}
	::= &  \ttkey{with associativity} {\em OPERATOR-ASSOCIATIVITY} \\
	$|$ &  \ttkey{left} \\
	$|$ &  \ttkey{right}
	\end{tabular} \\[1ex]
\key{OPERATOR-ASSOCIATIVITY} ::=
	\verb|[|
	\{ {\em word} $|$ {\em separator} \}
	\{ {\em word} $|$ {\em separator} \}$^\star$
	\verb|]| \\[1ex]
\key{operator-parser-option} ::=
	\ttkey{with parser} {\em PARSER-NAME} \\[1ex]
\ikey{PARSER-NAME}{parser-name} ::=
	\verb|[|
	\{ {\em word} $|$ {\em separator} \}
	\{ {\em word} $|$ {\em separator} \}$^\star$
	\verb|]| \\[1ex]
\key{operator-flags-option} \\
	\hspace*{0.5in}\begin{tabular}[t]{rl}
	::= &  \ttkey{with flags} {\em OPERATOR-FLAGS} \\
	$|$ &  \ttkey{definitional}
	\end{tabular} \\[1ex]
\key{OPERATOR-FLAGS} ::=
	\verb|[|
	\{ {\em word} $|$ {\em separator} \}
	\{ {\em word} $|$ {\em separator} \}$^\star$
	\verb|]| \\[1ex]
\key{operator-wrapper-option} ::=
	\ttkey{with wrapper} {\em OPERATOR-WRAPPER} \\[1ex]
\ikey{OPERATOR-WRAPPER}{operator-wrapper} ::=
	\verb|[|
	\{ {\em word} $|$ {\em separator} \}
	\{ {\em word} $|$ {\em separator} \}$^\star$
	\verb|]| \\[1ex]
\key{operator-subdefinitions-option} ::=
	\ttkey{with subdefinitions} {\em block}
\end{indpar}

which correspond to the definitions
(\secref{EXPRESSION-DEFINITIONS}):

\begin{indpar}
\begin{verbatim}
define qualifier with

define qualifier shortcuts
       ( [left => with associativity [left]],
         [right => with associativity [right]],
         [definitional => with flags [definitional]] )

define operator FIXITY NAME
		with precedence PRECEDENCE
		with associativity ASSOCIATIVITY
		with parser PARSER
		with flags FLAGS
		with wrapper WRAPPER
		with subdefinitions SUBDEFINITIONS
		<-- integer(PRECEDENCE)
\end{verbatim}
\end{indpar}

\subsubsection{Operator Selection}
\label{OPERATOR-SELECTION}

The operator selection algorithm is run by the
the \mkey{operators parser}{operator selection}
\ttmkey{-OPERATORS-PARSER-}{operator selection} to check an expression
for operators.
This algorithm selects operators whose name appears in the expression,
and then applies rules to de-select some selected operators.  If at the end
of the algorithm there is exactly one operator selected, the definition
of that operator is used to restructure the expression.  If there is more
than one operator selected, it is a parse error unless the definitions of
all selected operators are identical except for operator names.

The operator selection algorithm uses the current parsing definition stack
to determine which operators are defined.  The most important parts of each
operator definition are the operator name and operator fixity.

The operator selection algorithm produces a result that is usually
the same as the
more common context free grammar rules for parsing expressions.  However,
there is a difference: in PCASL an operator can change the parsing stack
used to parse its operand subexpressions.

In the operator selection algorithm the concepts of
prefix sequence and postfix sequence
are used.  A \key{prefix sequence} is a sequence of defined prefix operator
names.  A \key{postfix sequence} is a sequence of defined postfix operator
names.

The \key{operator selection algorithm} executes the following
steps in the order given, repeating each step until it does nothing,
before proceeding to the next step.

\begin{enumerate}

\item
Select any defined matchfix operator whose opening name
begins the expression and whose closing name ends the expression.

\item
If two matchfix operators are selected, the opening operator name
of the first is longer than the opening operator name of the second,
and the closing operator name
of the first is longer than the closing operator name of the second,
then the second operator is deselected.

Thus if `\verb/[| ... |]/' and `\verb/[ ... ]/' are two selected
matchfix operators, the second will be deselected.

\item
If more than one matchfix operator is still selected, the parse is in error.

\item
If a matchfix operator is selected, and the expression begins with
a prefix sequence that is not shorter than the opening name of the
matchfix operator, the parse is in error.
Similarly if a matchfix operator is selected, and the expression ends with
a postfix sequence that is not shorter than the closing name of the
matchfix operator, the parse is in error.

Thus if `\verb|- < ... > -|' is a defined matchfix operator and
`\verb|-|' is a defined prefix operator, but there is no defined
prefix operator whose name begins with the lexeme `\verb|<|',
parses of expressions containing the matchfix operator
will \underline{not} be in error.
But if there is another defined prefix operator named `\verb|<|',
parses of these expressions will be in error.

\item
If a matchfix operator is selected, the algorithm terminates successfully
at this point, without selecting any other operators.
The part of the expression between the opening operator name and closing
operator name is the sole operand of the matchfix operator.  This operand
is implicitly parenthesized, and may be empty.

\item
Select all defined infix operators that occur in the expression.

\item
De-select any selected infix operator that overlaps or abuts
a prefix sequence that begins the expression.
De-select any selected infix operator that overlaps or abuts
a postfix sequence that ends the expression.

Note that the empty sequence is both a prefix and a postfix sequence,
so infix operators that begin or end the expression are deselected.
For example, in the expression `\verb|+ x * y|' the prefix operator
`\verb|+|' is
deselected, and in the expression `\verb|+ + + x * y|' all the `\verb|+|'s are
deselected.

\item
On the set of selected infix operators define the \key{infix conflict relation}
to be the smallest equivalence relation such that two selected infix operators
conflict if they overlap or abut.

\item
De-select any infix operator if it is not followed by a prefix sequence that
contains all other infix operators that conflict with the infix operator.
De-select any infix operator if it is not preceded by a postfix sequence that
contains all other infix operators that conflict with the infix operator.
In checking for conflicting operators, check any infix operator selected
at the end of the previous step, whether or not it has been deselected
by repetitions of this step.

Since the empty sequence is a prefix and postfix sequence, any infix operator
that does not conflict with any other infix operator will still be selected
at the end of this step.

The parse is in error if there is an equivalence class of
conflicting operators all
of whose members are deselected by repetitions of this step.

\item
The parse is in error if two conflicting infix operators are still selected at
this point.

\item
If two infix operators are selected, and the first has strictly lower
precedence than the second, de-select the second.

\item
If several infix operators are still selected at this point, all these
operators must have the same associativity, which must not be `\verb|none|'.
Otherwise the parse is in error.

\item
If several infix operators are selected at this point and their common
associativity is `\verb|left|', all but the rightmost infix operator
are deselected.
Similarly if several infix operators are selected at this point and their common
associativity is `\verb|right|', all but the leftmost infix operator
are deselected.

\item
If several infix operators are still selected at this point, all these
operators must have the same definitions except for operator names.
Otherwise the parse is in error.

\item
If any infix operators are still selected at this point, the algorithm
terminates successfully.
The parts of the expression between the operators, before the first
operator, and after the last operator are the operands, and are
implicitly parenthesized.  No operand is empty.
If the common associativity of the infix operators is not
`\verb|left|',
`\verb|right|', or
`\verb|none|', this associativity as a lexeme sequence is prepended to the
expression, before the first operand.

\item
Select all prefix operators that begin the expression, and all postfix
operators that end the expression.

\item
If two prefix operators are selected, and one is longer than the other,
de-select the shorter.
Similarly, if two postfix operators are selected,
and one is longer than the other,
de-select the shorter.

\item
If both prefix and postfix operators are selected at this point, the postfix
operators are deselected.  Thus postfix operators are in effect
given precedence over prefix operators.

\item
If more than one operator is selected at this point, the parse is in error.
This would be the result of ambiguity among prefix operators, or ambiguity
among postfix operators.

\item
If one operator is selected at this point, and the operator name
is the entire expression, de-select the operator.  Thus
in the expression `\verb|x = (+)|', `\verb|+|' will be deselected
when the subexpression `\verb|(+)|' is parsed.

\item
The algorithm terminates successfully
at this point with zero operators selected,
or with one prefix or postfix operator selected.
If there is one selected operator, the part of the expression
that is not the operator name is the sole operand, and is implicitly
parenthesized.  Note this operand cannot be empty.

\end{enumerate}

As noted above, the parsing stack can be changed before subexpressions
are parsed, but the definitions of prefix, postfix, and infix operators
before the change affect operator selection.  It is possible for
parsing stack changes to produce anomalous seeming results.  For
example, if a parsing stack change introduced by the operator
\verb|+*+| defines a new prefix operator named \verb|*|,
where \verb|*| was previously as just an infix operator,
then the expression `\verb|x +*+ * y|' would be illegal, because
\verb|*| would not be defined as a prefix operator soon enough.
In this case one would have to write
`\verb|x +*+ (* y)|' to get a legal parse.


\subsubsection{Post Operator Selection Processing}
\label{POST-OPERATOR-SELECTION-PROCESSING}

Post operator selection processing is done after operator selection by
the \mkey{operators parser}{post operator selection}
\ttmkey{-OPERATORS-PARSER-}{post operator selection},
whether or not any operators are selected.
After operators are selected without a parse error, all the selected
operators have identical definitions except for operator name.
The common part of these definitions, which we will refer to as
`the operator definition', is used to control post operator selection
processing of the expression containing the operators.  If no operators are
selected, a definition containing no optional parts (no named associativity,
no parser, no control flags, no wrapper, no subdefinitions) is used
to control post operator selection processing.

Post operator selection processing is done in the following steps.

\begin{enumerate}

\item
If the operator definition has a parser, this is pushed onto the
parsing parser stack.

\item
If the operator definition has subdefinitions, these are pushed onto the
parsing definition stack.

\item
\label{PARSE-SUBEXPRESSIONS}
Subexpressions are parsed left to right by calling the parser at the
top of the parsing parser stack, and subexpressions are replaced by
their parsed versions.

An exception is made if the operator definition has the
\ttmkey{definitional}{control flag} control flag (note the operator must
be {\tt right} associative) and if the first
subexpression, after being parsed, is a definition that can be pushed onto the
parsing stack.  In this case, this definition
is pushed onto the parsing stack, the second subexpression is parsed,
the current expression is replaced by the parsed second expression,
and parsing continues at step~\ref{POP-PARSING-STACK} below.

\item
If the operator definition has no wrapper, it is processed as follows:

\begin{enumerate}

\item
If the operator definition has named associativity,
this associativity is prepended to the expression.

\item
Otherwise if the operator definition is not {\tt matchfix},
any selected operator (there can be at most one) is moved to the
beginning of the expression.

\item
Otherwise if the operator definition is {\tt matchfix}, the
parse is in error.

\end{enumerate}

\item
If the operator definition has a wrapper, it is processed as follows:

\begin{enumerate}

\item If the operator definition does not have named associativity,
all selected operators are removed from the expression.

\item
The expression is
replaced by its wrapper with the word \verb|...| in the wrapper
replaced by the expression.

\end{enumerate}

\item
\label{POP-PARSING-STACK}
If anything was pushed onto the parsing stack above, these things
are popped from the stack.

\item
The expression is returned as the result of parsing the expression.

\end{enumerate}

\subsection{Text Parsing}
\label{TEXT-PARSING}

\ikey{Text parsing}{text parsing} is performed by the \ttkey{-TEXT-PARSER-},
which is the parser for subexpressions of the
\verb|`|...\verb|'|,
\verb|``|...\verb|''|,
\verb|```|...\verb|'''|, etc. matchfix operators.
The \verb/|/ format separator and sentence and paragraph ends are
recognized by text processing, while
operators, qualifiers, qualifier shortcuts, and
the \verb|?:?|, \verb|<:>|, \verb|::>|, \verb|@@|, and
\verb|??| marks are \underline{not} recognized.

Text parsing is normally done in the context of a pair of matched
{\em quotes}, and in this context {\em white-space} pre-lexemes
become lexemes.  Note that {\em white-space} lexemes all consist
of zero or more {\em vertical space} characters followed by zero or
more {\em single-space} characters (see \secref{WHITE-SPACE-CONVERSION}).
There are three kinds of {\em white-space} lexemes
used by text parsing:

\begin{list}{}{}

\item[{\bf Spacer Lexemes}.]
A \key{spacer} lexeme is a {\em white-space} lexeme containing
only single spaces.  Spacers are used in text parsing if they follow
a \verb\|\ format separator on a line.

\item[{\bf Line Separator Lexemes}.]
A \key{line separator} lexeme is a {\em white-space}
lexeme that contains a single {\em line-feed} character and no other
{\em vertical-space} characters.  Such lexemes separate
non-blank lines, and are used by text parsing to end lines containing
a \verb\|\ format separator.

\item[{\bf Blank Line Lexemes}.]
A \key{blank line} lexeme is a {\em white-space}
lexeme that contains either two or more {\em line-feed} characters or contains
a {\em vertical-space} character that is not a {\em line-feed} character.
Such lexemes
represent blank lines between non-blank lines, and are used by text parsing
both to end lines containing a \verb\|\ format separator and
to separate paragraphs.

\end{list}

\subsection{Section, Paragraph, and Sentence Parsing}
\label{SECTION-PARAGRAPH-AND-SENTENCE-PARSING}

If the text being parsed does not contain any format separators,
the text is parsed phrases, sentences, and paragraphs.

First the text is divided by blank line lexemes into paragraphs.
The sequence of paragraphs comprises a section.

Then in each paragraph, sentence terminators are located.
White-space lexemes in the paragraph are deleted after sentence terminators
are located.
Each sequence
of lexemes or subexpressions ending in a sentence terminator
is made into a sentence, and any
non-empty sequence
of lexemes or subexpressions following the last sentence terminator
is made into a phrase.
The paragraph is then a sequence of zero or more sentences 
possibly followed by a phrase.

The syntax of the result is:

\begin{indpar}
\key{section} ::= \verb|[section| {\em paragraph} {\em paragraph}$^\star$
                  \verb|]| \\[1ex]
\key{paragraph}
	\begin{tabular}[t]{rl}
	::= &  \verb|[paragraph| {\em sentence} {\em sentence}$^\star$
	       \verb|]| \\
	$|$ &  \verb|[paragraph| {\em sentence}$^\star$ {\em phrase}
	       \verb|]|
	\end{tabular} \\[1ex]
\key{sentence} ::= \verb|[sentence| {\em sentence-non-terminator}$^\star$
                                    {\em sentence-terminator}
                  \verb|]| \\[1ex]
\key{phrase} ::= \verb|[phrase| {\em sentence-non-terminator}
                                    {\em sentence-non-terminator}$^\star$
                  \verb|]| \\[1ex]
\key{sentence-terminator} ::= \verb|.| $|$
                              \verb|?| $|$
                              \verb|!| \\[1ex]
\key{sentence-non-terminator} ::= {\em word} $|$
                                  {\em separator} $|$
                                  {\em subexpression}
\end{indpar}

There are several rules that modify the description just given:

{\bf Sentence Terminator Rule.}\index{Sentence Terminator Rule}
A \key{sentence-terminator} is any lexeme that is syntactically
a sentence terminator and is also not preceded by a {\em white-space}
lexeme.  All other lexemes and all subexpressions are
\skey{sentence-non-terminator}s.

{\bf Initial Capitalization Rule.}\index{Initial Capitalization Rule}
A {\em word} consisting of an initial capital letter followed
by zero or more lower case letters is converted
to an all lower case word if it begins a sentence or phrase.
A {\em word} consisting of an initial
\verb|^|\index{^@{\tt \Circumflex}} followed by an upper
case letter followed by zero or more lower case letters has the
initial \verb|^| removed.

{\bf Text Simplification Rule.}\index{Text Simplification Rule}
If the {\tt -TEXT-PARSER-} is to return a {\em section} with just
one {\em paragraph} and that {\em paragraph} contains nothing but
just one {\em sentence} or {\em phrase},
then just the {\em sentence} or {\em phrase} is returned.
Otherwise, if a {\em section} with just one {\em paragraph} is
to be returned, just the {\em paragraph} is returned.


Some examples follow:

\begin{center}
\begin{tabular}{lcl}

\verb|`the wife of Bob'|
& parses as &
\verb|[phrase the wife of Bob]|
\\[2ex]
\verb|`She hit the ball.'|
& parses as &
\verb|[sentence she hit the ball .]|
\\[2ex]
\begin{tabular}{@{}l@{}}
\verb|``^Bill swung.| \\
\verb|  But he missed!''|
\end{tabular}
& parses as &
\begin{tabular}{@{}l@{}}
\verb|[paragraph [sentence Bill swung .]| \\
\verb|           [sentence but he missed !]]|
\end{tabular}
\\[5ex]
\begin{tabular}{@{}l@{}}
\verb|``^I liked| \\
\verb|  the party.| \\
\verb|| \\
\verb|  Later, we caught| \\
\verb|  the bus.''|
\end{tabular}
& parses as &
\begin{tabular}{@{}l@{}}
\verb|[section| \\
\verb|   [paragraph [sentence I liked the]]| \\
\verb|                        party .]]| \\
\verb|   [paragraph [sentence later , we caught| \\
\verb|                        the bus .]]]| \\
\end{tabular}

\end{tabular}
\end{center}

Note that capitalized words like proper names and `\verb|I|' need to be
prefixed `\verb|^|' if they begin a sentence or phrase.


\section{Expression Graphs}
\label{EXPRESSION-GRAPHS}

Expression graphs store expressions.  An \key{expression graph}
is a directed possibly cyclic graph with labels on all arrows and on some
nodes, with a designated root node, and with a possibly empty
set of graph node valued variables.

Expression graph arrow and node labels 
must be single words or separators.  Labeled nodes cannot be the source
of any arrow.  Several arrows with the same source node may have the same
arrow label.

A node that does not have a label and is not the source of any arrows
is called a \key{null node}.\label{NULL-NODE}

An expression graph variable has a name, a value, and an optional default.
Variable names must be single words or separators.
Both the value and the default are graph nodes.  The default may be
missing.  Variable values are often, but by no means always, null nodes.

All nodes in an expression graph are reachable from either the root or
from the value or default of one of the graph's variables.

If there are no variables and no node is the destination
of more than one arrow, the graph is a tree.  In this case it is called
an \key{expression tree}.

There is a correspondence between raw expressions and certain expression
trees, called \skey{raw expression tree}s,
that is one-to-one, except that two raw expressions differing only
in the order of their qualifier clauses and the order of their reorderable
arguments correspond to the same expression tree.

There are several operations that convert raw expression trees into general
expression graphs (\secref{EXPRESSION-GRAPH-REPRESENTATION}).
All these use designated operators, such as
`\verb|//|', to identify expression graph variables and their values.
The most commonly used conversion operation also uses the convention that
words beginning with a capital letter or with `\verb|#|' followed by a
capital letter name expression graph variables.

\subsection{Expression Graph Notation}
\label{EXPRESSION-GRAPH-NOTATION}

The following is an example of the notation we will use to represent
expression graphs:%
\footnote{This notation is standard for feature structures:
e.g., see Bob Carpenter, {\em The Logic of Typed Feature Structures},
1992, Cambridge University Press}

\newcommand{\Glabel}[1]{\mbox{\tt #1}}
\newcommand{\Gvariable}[1]{\fbox{\tt #1}}
\newsavebox{\Gfbox}
\newcommand{\Gdefault}[1]{%
    \sbox{\Gfbox}{\fbox{\tt #1}}%
    {\setlength{\fboxsep}{0.01in}%
     \fbox{\usebox{\Gfbox}}%
    }}
\newenvironment{Graph}%
    { \( \left[ \begin{array}{l}}%
    {\end{array} \right] \) }
\newenvironment{Gchild}[1]%
    {\mbox{\tt #1}:
     \begin{array}{l}
     \rule{0in}{0in} \vspace{-0.15in} \\
     \left[ \begin{array}{l}}%
    {\end{array} \right] \\
     \rule{0in}{0in} \vspace{-0.15in}
     \end{array}}
% Gsingle is Gchild with only one thing in its brackets so
% the brackets can be eliminated.
\newenvironment{Gsingle}[1]%
    {\mbox{\tt #1}:
     \begin{array}{l}
     \rule{0in}{0in} \vspace{-0.15in} \\
     \begin{array}{l}}%
    {\end{array} \\
     \rule{0in}{0in} \vspace{-0.15in}
     \end{array}}

\begin{center}

\begin{Graph}
    \Gvariable{\CurlyBra ROOT\CurlyKet} \\
    \begin{Gchild}{fee}
	\Gvariable{X} \\
	\Glabel{51} \\
    \end{Gchild} \\
    \begin{Gchild}{fie}
	\Gvariable{X} \\
    \end{Gchild} \\
\end{Graph}
\hspace{0.3in}
\begin{Graph}
    \Gdefault{X} \\
    \begin{Gchild}{1}
	\Glabel{foe} \\
    \end{Gchild} \\
    \begin{Gchild}{2}
	\Glabel{fum} \\
    \end{Gchild} \\
\end{Graph}
\hspace{0.3in}
\begin{Graph}
    \Gvariable{Y} \\
    \Gdefault{Y} \\
    \begin{Gchild}{1}
	\Gvariable{\protect\CurlyBra 1\protect\CurlyKet} \\
    \end{Gchild} \\
    \begin{Gchild}{2}
	\begin{Gchild}{*}
	    \Gvariable{\protect\CurlyBra 1\protect\CurlyKet} \\
	\end{Gchild} \\
    \end{Gchild} \\
\end{Graph}


\end{center}

In this notation, the representation of a graph node is a set of lines
enclosed in [~] brackets.

If the node is the value or default of some variables,
the names of those variables, each enclosed in a box or double box, are the top
lines of the representation of the node.
A variable name is boxed if the node is the value of
a variable, and double boxed if the node is the default of the
variable.
In our example, \Gvariable{X}, and \Gvariable{Y}
designate nodes that are variable values and
\Gdefault{X} and \Gdefault{Y} designate nodes that are variable defaults.
The \key{root node} is treated
as if it were the value of a variable named `\verb|{ROOT}|',
and is therefore designated by \Gvariable{\CurlyBra ROOT\CurlyKet} in our
example.
A single node can be the value or default of several variables: in the
example one node is both the value and the default of \verb|Y|.

The example expression graph has two variables, \verb|X|
and \verb|Y|.  While all nodes in the expression graph must be reachable
from the root or from the value or default of some variable in the graph,
all nodes need not be reachable from the root,  The above example
is a single expression graph with a root and two variables, and \underline{not}
three expression graphs.

Two nodes with the same boxed name in their representation
are actually the same node in the graph, and
similarly two nodes with the same doubly boxed name in their
representation are actually the same node in the graph.  When several
nodes are the same in this sense, all but one of these must have only
boxed or doubly boxed names in its representation.

Sometimes it is necessary to indicate that two nodes are the same for
a node that is not the root or the value or default of any variable.
This is done by creating an
\key{expression graph pseudo-variable}\index{pseudo-variable!expression graph}%
\label{PSEUDO-VARIABLES}
whose name is a word or separator in curly brackets.  In the example,
\verb|{1}| is such a pseudo-variable name, and of course \verb|{ROOT}|
is a pseudo-variable name.  Pseudo-variable names are always boxed, and
never double boxed.

If a node has a label, that is the last line in the node's representation,
and is
immediately below any boxed or doubly boxed names.
There are no other lines in the representation.
In the above example, \Glabel{51} labels
the node that is the value of the variable \verb|X|.

If a node is the source of arrows in the directed graph,
representations of each arrow appear below any boxed or doubly
boxed variable names in the
representation of the node.  Each arrow representation consists of the
label of the arrow followed by a colon (:) followed by
a representation of the target node of the arrow.  This later
is of course surrounded by [] brackets.

It is permitted to omit the [] brackets when only one thing is
bracketed.  Thus the above example can also be rendered as:

\begin{center}

\begin{Graph}
    \Gvariable{\CurlyBra ROOT\CurlyKet} \\
    \begin{Gchild}{fee}
	\Gvariable{X} \\
	\Glabel{51} \\
    \end{Gchild} \\
    \begin{Gsingle}{fie}
	\Gvariable{X} \\
    \end{Gsingle} \\
\end{Graph}
\hspace{0.3in}
\begin{Graph}
    \Gdefault{X} \\
    \begin{Gsingle}{1}
	\Glabel{foe} \\
    \end{Gsingle} \\
    \begin{Gsingle}{2}
	\Glabel{fum} \\
    \end{Gsingle} \\
\end{Graph}
\hspace{0.3in}
\begin{Graph}
    \Gvariable{Y} \\
    \Gdefault{Y} \\
    \begin{Gsingle}{1}
	\Gvariable{\protect\CurlyBra 1\protect\CurlyKet} \\
    \end{Gsingle} \\
    \begin{Gsingle}{2}
	\begin{Gsingle}{*}
	    \Gvariable{\protect\CurlyBra 1\protect\CurlyKet} \\
	\end{Gsingle} \\
    \end{Gsingle} \\
\end{Graph}


\end{center}

\subsection{Pure Unification}
\label{PURE-UNIFICATION}

Informally, expression graph nodes represent expressions that
may be thought of as containing information, and the unification of
two nodes is then a node whose information content
is the least upper bound of the information content of each of
the two nodes taken separately.

PCASL uses two forms of expression graph node unification.
The pure form, called `\key{pure unification}',
is just the standard labeled graph unification
algorithm, and is specified in this section.  The other form, which
has a special merge operation for nodes representing calls on
functions or builtin PCASL operations, is called `call unification',
and is described in \Secref{CALL-UNIFICATION}.

Unification of two expression graph nodes
either succeeds and produces a unique expression
graph node as its result, or fails and produces no result.

Unification algorithms use a set called the \key{merge set}
of node pairs called \skey{merge pair}s.  For pure unification of
two nodes, X and Y, this is initialized to the set containing
only the one merge pair, (X,Y).

A unification algorithm simply extracts a merge pair (M,N) from the
merge set, merges the two nodes named, M and N, and does this
repeatedly until the merge set is empty.  The algorithm succeeds if
every pair of nodes that must be merged can be merged.  When two nodes
are merged, they become the same node.  Merging two nodes can
create new merge pairs.  However, merging two nodes that are identical
does nothing, and since expression graph memory is finite,
the algorithm must terminate after all nodes are merged, if it does not
terminate before then.

In merging two nodes M and N, one of these nodes, say M, is replaced by
a forwarding pointer to the other node, say N, so every reference to M
becomes a reference to N.  N may also be modified by giving it a label
or creating arrows that it sources.  And as mentioned above, new merge
pairs may be created.

If a unification algorithm is unsuccessful, all changes to expression
graph memory are undone.

For pure unification of two nodes X and Y, the merge set is initially
the single pair (X,Y).  If unification is successful, X and Y are merged,
and the result of the unification is the merged node, i.e., the node
X, or equivalently the node Y.

It is possible to run unification with arbitrary initial merge sets.

For pure unification the \key{node merge algorithm}%
\label{NODE-MERGE-ALGORITHM}
applied to the merge pair (M,N) is as follows:

\begin{enumerate}

\item
Remove (M,N) from the merge set.

\item
If M and N are the same node, terminate the node merge algorithm.

\item
\label{UNIFICATION-LABEL-FAILURE}
If M and N are labeled nodes with different labels, the unification
algorithm fails.

\item
If either M or N is a labeled node and the other node is the
source of some arrows, the unification algorithm fails.

\item
\label{UNIFICATION-AMBIGUITY-FAILURE}
If there is an arrow label L such that one of the nodes M or N
is the source of two or more arrows labeled L and the other node is the
source of one or more arrows labeled L, the unification algorithm fails.
Intuitively, when we try to match the arrows sourced at one of the
two nodes with arrows sourced at the other of the two nodes so
that matched arrows have the same arrow label, then we must get a
unique answer.

\item
If M is a labeled node with label L, and N is unlabeled, label N with L.

\item
For every arrow sourced at M do the following.
Let the arrow be labeled L and point at target node MT.
If N is not the source of an arrow labeled L,
make a new arrow labeled L pointing from N to MT.
But if N is the source of an arrow labeled L with target NT, add the
merge pair (MT,NT) to the merge set.

\item
Replace M by a forwarding pointer that points at N.
This means that henceforth any attempt to reference M will be forwarded
to N, so M is effectively merged with N.

\end{enumerate}

Note that the running time of the pure unification algorithm is bounded by

\[
\begin{array}{c}
T(\mbox{number of nodes reachable from nodes in the initial merge set}) \\
\times \\
(\mbox{number of arrows sourced at these nodes})^2
\end{array}
\]

where $T$ is a small constant time (typically a few microseconds).
The total number of nodes reachable from nodes in the initial merge set
bounds the number of node merge
operations, and the
last factor bounds the time to form the union of the arrow labels of
any two nodes that are being merged.  The bound just given is typically
a substantial overestimate, and can be improved to:

\[
\begin{array}{c}
T \left(
   \begin{array}{c}
   \mbox{number of nodes reachable from nodes in the initial merge set} \\
    - \\
  \mbox{number of nodes reachable from these nodes after unification} \\
  \end{array}
  \right) \\
\times \\
(\mbox{maximum number of arrows sourced at any one node reachable
       after unification})^2
\end{array}
\]

\subsection{Raw Expression Trees}
\label{RAW-EXPRESSION-TREES}

Raw expressions represent certain expression trees called
\skey{raw expression tree}s.  For example, the expression

\begin{center}
\verb|[max x y z]|
\end{center}

represents the raw expression tree

\begin{center}
\begin{Graph}
    \Gvariable{\CurlyBra ROOT\CurlyKet} \\
    \begin{Gsingle}{-ARITY-}
	\Glabel{3} \\
    \end{Gsingle} \\
    \begin{Gsingle}{0}
	\Glabel{max} \\
    \end{Gsingle} \\
    \begin{Gsingle}{1}
	\Glabel{x} \\
    \end{Gsingle} \\
    \begin{Gsingle}{2}
	\Glabel{y} \\
    \end{Gsingle} \\
    \begin{Gsingle}{3}
	\Glabel{z} \\
    \end{Gsingle} \\
\end{Graph}
\end{center}

The raw expression is represented by the root node of a raw expression tree.
The raw expression head, `\verb|max|', is represented by a node labeled
`\verb|max|' that is the target of an arrow labeled `\verb|0|' from the root.
Here `\verb|0|' is called the \key{head index} of the root node, the
arrow it labels is called the \key{head arrow} of the root node,
and the target of that arrow is called the \key{head} of the root node.

The three arguments become targets of three arrows from the root labeled
`\verb|1|', `\verb|2|', and `\verb|3|'.
Each argument in this case is a single
word, and is represented by a node labeled with that word.
Here non-zero natural numbers are
\ikey{un-reorderable argument indices}{un-reorderable argument index} of the
root that label
\skey{un-reorderable argument arrow}s of the root, and the targets of these
arrows are the \skey{un-reorderable argument}s of the root.  Reorderable
and rest arguments are introduced below.

An arrow labeled `\ttkey{-ARITY-}' is added to the root node to point
at a node labeled with the number of arguments in the raw expression.
This arrow is called the \key{arity arrow} of the root node,
its target is called the \key{arity target} of the root, and
the label of that target is called the \key{arity} of the root.

A second example is the expression

\begin{center}
\verb|[sort x @@ with key weight @@ with comparator <]|
\end{center}

which is represented by the raw expression tree

\begin{center}
\begin{Graph}
    \Gvariable{\CurlyBra ROOT\CurlyKet} \\
    \begin{Gsingle}{-ARITY-}
	\Glabel{1} \\
    \end{Gsingle} \\
    \begin{Gsingle}{0}
	\Glabel{sort} \\
    \end{Gsingle} \\
    \begin{Gsingle}{1}
	\Glabel{x} \\
    \end{Gsingle} \\
    \begin{Gchild}{-REQUIRED-QUALIFIER-}
	\begin{Gsingle}{-ARITY-}
	    \Glabel{2}
	\end{Gsingle} \\
	\begin{Gsingle}{0}
	    \Glabel{with}
	\end{Gsingle} \\
	\begin{Gsingle}{1}
	    \Glabel{key}
	\end{Gsingle} \\
	\begin{Gsingle}{2}
	    \Glabel{weight}
	\end{Gsingle} \\
    \end{Gchild} \\
    \begin{Gchild}{-REQUIRED-QUALIFIER-}
	\begin{Gsingle}{-ARITY-}
	    \Glabel{2}
	\end{Gsingle} \\
	\begin{Gsingle}{0}
	    \Glabel{with}
	\end{Gsingle} \\
	\begin{Gsingle}{1}
	    \Glabel{comparator}
	\end{Gsingle} \\
	\begin{Gsingle}{2}
	    \Glabel{<}
	\end{Gsingle} \\
    \end{Gchild} \\
\end{Graph}\label{QUALIFIER-EXAMPLE}
\end{center}

Here each qualifier becomes the target of an arrow labeled
`\ttkey{-REQUIRED-QUALIFIER-}' because
the `\verb|@@|' \key{required-qualifier-mark} is used (the arrow label
would be `{\tt -OPTIONAL-QUALIFIER-}' if the `\verb|??|'
optional-qualifier-mark had been used).
The arrows with these labels are called \skey{qualifier arrow}s of the
root node and their targets are called \skey{qualifier}s of the root.
In this case there are two \skey{required qualifier arrow}s and
two \skey{required qualifier}s.
Each qualifier sources arity, head, and un-reorderable argument arrows
analogous to those of the root node, and could source qualifier,
reorderable argument (see below), and rest argument (see below) arrows.

A third example is the expression

\begin{center}
\verb|[sort x @@ with key weight ?? with order X]|
\end{center}

which is represented by the raw expression tree

\begin{center}
\begin{Graph}
    \Gvariable{\CurlyBra ROOT\CurlyKet} \\
    \begin{Gsingle}{-ARITY-}
	\Glabel{1} \\
    \end{Gsingle} \\
    \begin{Gsingle}{0}
	\Glabel{sort} \\
    \end{Gsingle} \\
    \begin{Gsingle}{1}
	\Glabel{x} \\
    \end{Gsingle} \\
    \begin{Gchild}{-REQUIRED-QUALIFIER-}
	\begin{Gsingle}{-ARITY-}
	    \Glabel{2}
	\end{Gsingle} \\
	\begin{Gsingle}{0}
	    \Glabel{with}
	\end{Gsingle} \\
	\begin{Gsingle}{1}
	    \Glabel{key}
	\end{Gsingle} \\
	\begin{Gsingle}{2}
	    \Glabel{weight}
	\end{Gsingle} \\
    \end{Gchild} \\
    \begin{Gchild}{-OPTIONAL-QUALIFIER-}
	\begin{Gsingle}{-ARITY-}
	    \Glabel{2}
	\end{Gsingle} \\
	\begin{Gsingle}{0}
	    \Glabel{with}
	\end{Gsingle} \\
	\begin{Gsingle}{1}
	    \Glabel{order}
	\end{Gsingle} \\
	\begin{Gsingle}{2}
	    \Glabel{X}
	\end{Gsingle} \\
    \end{Gchild} \\
\end{Graph}
\end{center}

This is similar to the last example except one qualifier is an
`\ttkey{-OPTIONAL-QUALIFIER-}' because the
`\verb|??|' \key{optional-qualifier-mark} is used.
In this case there is one \skey{required qualifier arrow},
one \skey{required qualifier}, one \key{optional qualifier arrow}, and
one \key{optional qualifier}.  The only difference between
required and optional qualifiers is the label on the arrow to the qualifier
node, which is either `\ttkey{-REQUIRED-QUALIFIER-}'
or `\ttkey{-OPTIONAL-QUALIFIER-}'.

A fourth example is the expression

\begin{center}
\verb|[fill Prefix ?:? Matrix <:> Value Postfix ::> Runs]|
\end{center}

which is represented by the raw expression tree

\begin{center}
\begin{Graph}
    \Gvariable{\CurlyBra ROOT\CurlyKet} \\
    \begin{Gsingle}{-ARITY-}
	\Glabel{1-4} \\
    \end{Gsingle} \\
    \begin{Gsingle}{0}
	\Glabel{fill} \\
    \end{Gsingle} \\
    \begin{Gsingle}{1}
	\Glabel{Prefix} \\
    \end{Gsingle} \\
    \begin{Gsingle}{2-3}
	\Glabel{Matrix} \\
    \end{Gsingle} \\
    \begin{Gsingle}{2-3}
	\Glabel{Value}
    \end{Gsingle} \\
    \begin{Gsingle}{4}
	\Glabel{Postfix}
    \end{Gsingle} \\
    \begin{Gsingle}{-REST-}
	\Glabel{Runs}
    \end{Gsingle} \\
\end{Graph}
\end{center}

Arguments appearing after the \key{optional argument mark} \ttkey{?:?} are
optional.  This is indicated by the arity `\verb|1-4|' that indicates
that argument `\verb|1|' is the last required argument and there are
`\verb|4|' arguments total.  In general, the arity can have the form
`$m$\verb|-|$n$', where $0\leq m<n$, $m$ is the
\skey{number of required argument}s, $n$ is the
\skey{total number of argument}s,%
\label{TOTAL-NUMBER-OF-ARGUMENTS}
and $n-m$ is the number of optional arguments.

If there are no optional arguments and the arity is just a natural
number $n$, then $n$ is both the number of required arguments and the
total number of arguments.

Consecutive arguments separated by the \key{reorder mark} \ttkey{<:>} are
reorderable.  These arguments are indicated by argument indices that
describe the positions the arguments can occur in.  Thus in the example
the two arguments with index `\verb|2-3|' can each appear as either the
2nd or 3rd argument.  In general,
an argument index that is a range of the form
`$i$\verb|-|$j$', where $1\leq i<j\leq n$, and
$n$ is the total number of arguments, indicates that the argument can
be the $i$'th through $j$'th argument position.
These argument indices are called the
\ikey{reorderable argument indices}{reorderable argument index} of the
root node,
the arrows they label are called \skey{reorderable argument arrow}s
of the root,
and the targets of these arrows are called \skey{reorderable argument}s
of the root.

Note that if an argument has index `$i$\verb|-|$j$', then the $i$'th
and $j$'th argument must be either both required or both optional.
That is, if there are optional arguments and
$m$ is the number of required arguments, either
$j\leq m$ or $m<i$.

The argument after a \key{remainder mark} \ttkey{::>} is
given the special argument index `\ttkey{-REST-}' and names a list of
all arguments that occur after the total number of arguments specified
by the arity.  Here \verb|-REST-|
is called the \key{rest argument index} of the root node, the arrow it labels
is called the \key{rest argument arrow} of the root, 
and the target of this arrow is called the \key{rest argument}.
The rest argument is not counted in the arity.

\subsection{Call Nodes and the Call Check}
\label{CALL-NODES-AND-CHECK}

During expression evaluation (\secref{EXPRESSION-EVALUATION})
the expression being evaluated is unified with a pattern in an expression
definition.  The unification used for this purpose is
a variant called `call unification' (\secref{CALL-UNIFICATION}).

Call unification treats graph nodes called `call nodes'
differently than pure unification (\secref{PURE-UNIFICATION}) does.
A \key{call node} is a node that has
an \mkey{arity arrow}{of call node}.
That is, it is the source of an arrow labeled
\ttmkey{-ARITY-}{of call node}.  Call unification
replaces the node merge algorithm
(\pagref{NODE-MERGE-ALGORITHM})
with the call merge algorithm described below if one of the nodes
being merged is a call node and the other is not a
null node.

The call merge algorithm performs a subalgorithm called
the \key{call check algorithm} that checks whether a call node
is legal.  Specifically, a call node N passes the call check
algorithm if and only if the following are true:

\begin{enumerate}

\item
The arity of N is either a natural number $n$ (string of digits without
high order zeros) or has the form $m$\verb|-|$n$ where $m$ and $n$
are natural numbers and $m<n$.  In the second case $m$ is called the
\key{minimum arity} of N and $n$ is called the
\key{maximum arity} of N.  In the first case $n$ is both the minimum
and the maximum arity of N.

\item
N has exactly one head arrow.

\item
There exists a 1-1 map between the set of un-reorderable and reorderable
(but not rest) argument arrows of N and the set of
natural numbers from \verb|1| through $n$, where $n$ is the
maximum arity of N, such that if $i$ is a natural number
from \verb|1| through $n$, the arrow mapped to $i$
is either an un-reorderable argument arrow labeled $i$, 
or is a reorderable argument arrow labeled $j$\verb|-|$k$
where $j\leq i\leq k$.

Such a mapping is called a \key{argument order assignment} of N.%
\label{ARGUMENT-ORDER-ASSIGNMENT}
If there are any reorderable arguments of N, there will be
more than one argument order assignment of N.

\item
\label{NULL-REST-ARGUMENT}
N has at most one rest argument arrow.  If N has a rest argument
arrow, the rest argument of N is a null node (a node with no label
that is the source of no arrows).

\item
N can have any number of qualifier arrows (required or optional).

\item
There are no arrows sourced at N other than those with labels
enumerated above.  That is, any arrow sourced at N is either an arity
arrow, a head arrow, an argument arrow (un-reorderable, reorderable, or rest),
or a qualifier arrow (required or optional).

\item
The qualifiers of N (targets of qualifier arrows)
pass the call check algorithm recursively.  The call check
algorithm allows cycles in the graph, e.g. the case that N and one
of its qualifiers are the same node.

Note that raw expressions cannot represent call nodes that have
qualifiers which in turn have qualifiers, but the conversions
of \Secref{EXPRESSION-GRAPH-REPRESENTATION} can represent such call nodes.
Because of the difficulty of representing them, such call nodes are
of mostly theoretical interest.

\end{enumerate}

\subsection{Paths and Witnesses}
\label{PATHS-AND-WITNESSES}

A \ikey{non-empty path}{path!non-empty}
is a sequence of one or more arrows in some expression graph,
such that if A1 and A2 are consecutive
arrows in the sequence, the target of A1 is the source of A2.
The \mkey{path name}{of non-empty path}
of the path is the
sequence of words and separators that are the arrow labels of the
arrows of the path.  The source node of the first arrow in the path
is the \mkey{source}{of non-empty path} of the path.
The target node of the last
arrow in the path is the \mkey{target}{of non-empty path} of the path.
Note that the path name and source node of a path are together
not adequate to determine the path in all cases, since we allow
a node to source several arrows with the same label.

An \ikey{empty path}{path!empty} is just a node.  This node is both
the \mkey{source}{of empty path} of the empty path and
the \mkey{target}{of empty path} of the empty path.
The \mkey{path name}{of empty path} of the path is the empty sequence.

A \key{path name} is just a possibly empty sequence of words or separators.
A \key{node label} is just any word or separator.

A \key{witness} is a path name and a node label.
We use `{\em path-name}\,: {\em node-label}\,' to name the witness.
A \mkey{witness}{of node} of a node S in an expression graph is a witness
P:~L with path name P and node label L such that there is a path with source S,
path name P, and target that has node label L.

For example, consider the expression graph of

\begin{center}
\verb|[sort x @@ with key weight @@ with comparator <]|
\end{center}

on \pagref{QUALIFIER-EXAMPLE}.  The root of this expression graph
has `\verb|0|: \verb|sort|' as one of its witnesses and
`\verb|-REQUIRED-QUALIFIER- 0|: \verb|with|' as another.  There are
actually two distinct paths corresponding to this second witness.

Two witnesses are said to be \mkey{incompatible}{witness}
if they have the same path name but have different labels.
Two witnesses are said to be \mkey{compatible}{witness}
if they are not incompatible: that is, if they have different path
names, or if they have both the same path names and the same labels.
For example, `\verb|0|: sort' and `\verb|1|: x' are compatible while
`\verb|0|: sort' and `\verb|0|: switch' are incompatible.

It is not true that all witnesses for a node in an expression graph
are compatible.  This is because we allow one node to source several
arrows with the same label.  In the example expression graph
on \pagref{QUALIFIER-EXAMPLE},
`\verb|-REQUIRED-QUALIFIER- 1|: \verb|key|' and
`\verb|-REQUIRED-QUALIFIER- 1|: \verb|comparator|' are incompatible
witnesses for the value of \verb|{ROOT}|.  However, for an expression
graph in which no node sources two arrows with the same arrow label, it can
be easily proved that all witnesses of every node are compatible.

Two expression graph nodes, S1 and S2, possibly the same node, and possibly
in different expression graphs,
are said to be \key{witness compatible} if and only if every
witness for S1 is compatible with every witness for S2.  Otherwise
S1 and S2 are said to be \key{witness incompatible}.

For an integer N$\geq$0, an N-witness is a witness whose path name
has at most N arrow labels.
Two expression graph nodes, S1 and S2,
are said to be \key{N-witness compatible} if and only if every
N-witness for S1 is compatible with every N-witness for S2.  Otherwise
S1 and S2 are said to be \key{N-witness incompatible}.

It is possible for
a node to be 1-witness incompatible with itself, without being witness
incompatible with many other nodes.  For example, a node whose only
witnesses are the incompatible witnesses `\verb|1|:~\verb|fee|' and
`\verb|1|:~\verb|fie|' would be witness compatible with any
node that was not the source of any arrow with arrow label `\verb|1|'.

It can be easily proved that if two expression graphs can be unified
by pure unification (\secref{PURE-UNIFICATION}),
then any two nodes that are merged during
unification are 1-witness compatible.  This is because
step \stepref{UNIFICATION-AMBIGUITY-FAILURE} of the node merge algorithm
(\pagref{NODE-MERGE-ALGORITHM})
means that if the nodes have incompatible 1-witnesses then the nodes each
source only one arrow with the witness arrow label and therefore
the targets of these two arrows must merge which is forbidden by
step \stepref{UNIFICATION-LABEL-FAILURE}.

It can even be proved that if two expression graphs can be unified
by pure unification (\secref{PURE-UNIFICATION}),
then any two nodes that are merged during
unification are witness compatible.  This is because
if there are two incompatible witnesses of the two nodes, there
would be two paths with the same path name,
one for each witness, with consecutive
nodes on the two paths being merged, because
step \stepref{UNIFICATION-AMBIGUITY-FAILURE}
requires that along each of the two paths each node sources exactly
one arrow with the next arrow label on the paths.
The two paths would end at two nodes that
would have different labels, but would have to be merged, causing
the unification algorithm to fail by step \stepref{UNIFICATION-LABEL-FAILURE}.

In the unification algorithm,
arrow labels are used to match arrows to be merged.  More specifically,
when two nodes M and N are merged, arrows sourced at M are matched
with arrows sourced at N by arrow label.  If this match cannot be done,
step \stepref{UNIFICATION-AMBIGUITY-FAILURE} of the unification algorithm
causes unification to fail.  In order to merge qualifier nodes,
we need to avoid this failure, and we do this in \Secref{CALL-UNIFICATION}
by using some of the
the 1-witnesses of the qualifier nodes to make the match.
The 1-witnesses we use include the qualifier head, if that
is a node with a label, and any un-reorderable non-rest
qualifier argument that is simply a node with a label.

\subsection{Call Unification}
\label{CALL-UNIFICATION}

\ikey{Call unification}{call unification}\index{unification!call}
is a variant of pure unification (\secref{PURE-UNIFICATION}) that differs
from pure unification when merging nodes one of which is a call node
and the other of which is not a null node (node with no label that sources
no arrows).  Call unification is
used in the expression evaluation algorithm (\secref{EVALUATION-ALGORITHM}).

Call unification is identical to pure unification (\secref{PURE-UNIFICATION})
with two differences.  The first difference is that when one of two nodes
being merged is a call node (\secref{CALL-NODES-AND-CHECK}), and the other
node is not a null node, call unification
replaces the node merge algorithm (\pagref{NODE-MERGE-ALGORITHM})
with the call merge algorithm described below.

The second difference is that call unification makes choices during its
execution, because the call merge algorithm makes choices.  Each
set of possible choices over the whole call unification algorithm
leads to a separate algorithm execution that either succeeds or fails.

The \key{call merge algorithm} for a merge pair (M,N) in the merge set,
where either M or N or both is a call node, and neither is a null node,
with a particular choice, is as follows.

\begin{enumerate}

\item
Remove (M,N) from the merge set.

\item
If either M or N fails to pass a call check, the unification algorithm
fails.

\item
If M and N are the same node, terminate the call merge algorithm.

\item
A call node is said to be \key{indeterminate} if it has
an optional, reorderable, or rest argument or an optional qualifier.
Otherwise the node is said to be \key{determinate}.

If \underline{both} M and N are indeterminate,
the unification algorithm fails.

Otherwise, if just N is indeterminate, switch M and N so that
N is always determinate.

Later in this call merge algorithm M will be replaced by a forwarding
pointer to N, and N will not be given any optional, rest, or reorderable
arguments, so the result of the merge will be determinate.

\item
If M is determinate and the arities of M and N are not equal,
the unification algorithm fails.

If M has more required arguments than N has arguments, the unification
algorithm fails.

\item
For the rest of this algorithm,  let MT be the total number of arguments
(\pagref{TOTAL-NUMBER-OF-ARGUMENTS})
of M and NT be the total number of arguments of N.

\item
If M has a rest argument R, then
For each natural number I from MT$+$1 through NT,
an arrow labeled I$-$MT is added that has R as its source and the
I'th argument of N as its target.
Note that before this is done R is a null node
(step \stepref{NULL-REST-ARGUMENT} of the call check algorithm on
\pagref{NULL-REST-ARGUMENT}).
If NT$\leq$MT, this step does nothing (R remains a null node).

\item
Choose an argument order assignment (\pagref{ARGUMENT-ORDER-ASSIGNMENT})
for M.  For \verb|1|$\leq$I$\leq$MT, let MI let the assigned
I'th argument of M.

\item
Let T be the minimum of MT and NT.
For \verb|1|$\leq$I$\leq$T, let
NI be the I'th argument of N,
and add the merge pair (MI,NI) to the merge set.
For T$<$I$\leq$MT (if T$<$MT), add an arrow labeled I from N to MI.

\item
Let MH be the head of M and NH be the head of N.  Add a merge pair
(MH,NH) to the merge set.

\item
If there is any qualifier arrow of M whose target is
is 1-witness compatible with the target of more than one qualifier arrow
of N, the unification algorithm fails.  Ditto with M and N switched:
if there is any qualifier arrow of N whose target
is 1-witness compatible with the target of more than one qualifier arrow
of M, the unification algorithm fails.

\item
For each qualifier arrow MA of M whose target is 1-witness compatible with
the target of exactly one qualifier arrow NA of N, let MQ be the target
of MA and NQ be the target of NA.  Add (MQ,NQ) to the merge set.

\item
For each qualifier arrow MA of M whose target is 1-witness compatible with
the targets of no qualifier arrows N, let MQ be the target of MA,
and make an arrow from N to MQ
that has the arrow label \verb|-REQUIRED-QUALIFIER|.

\item
Replace M by a forwarding pointer that points at N.
This means that henceforth any attempt to reference M will be forwarded
to N, so M is effectively merged with N.

\end{enumerate}

\subsection{Expression Graph Representation}
\label{EXPRESSION-GRAPH-REPRESENTATION}

Raw expressions can be used to represent expression graphs.  This is
done by defining \skey{graph creation conversion}s that convert
raw expression trees to expression graphs.

A graph creation conversion is performed
by first converting the raw expression to an expression tree plus
a set of variables where each variable has not a single value or
single optional default, but instead has a set of one or more values
and zero or more defaults.  Then all the values of each variable
are unified using pure unification, and all the defaults of each
variable with defaults are similarly unified.  The result after
unification is an expression graph.

There are two kinds of graph creation conversion: explicit and implicit.
In implicit conversion words beginning with a capital letter or
with `\verb|#|' followed
by a capital letter are taken to be variable names if they appear where
a node label would appear.  In explicit conversion such words are not
given special treatment.

In converting a raw expression to an expression tree,
the rules for representing raw expressions as expression trees
(\secref{RAW-EXPRESSION-TREES})
are followed, with the following exceptions:

\begin{indpar}
\begin{list}{}{}

\item[{\tt [}{\tt //} {\em variable-name} {\em variable-value}{\tt]}]~\\
This expression converts to the same tree that the {\em variable-value}
converts to, and adds the root node of this tree to the value set of
the named variable.

\item[{\tt [}\ttkey{//} {\em variable-name}{\tt]}]~\\
This expression converts to a tree consisting of a single null root
node, and adds this node to the value set of the named variable.

\item[{\em variable-name}]~\\
For implicit conversion this is
treated as the equivalent of `\verb|[// |{\em variable-name}\verb|]|'
if {\em variable-name} begins with a capital letter or with
`\verb|#|' followed by a capital letter.
Otherwise this converts to a tree with a single node
labeled with a word that is not in general a variable name.

\item[{\tt [\Tilde\Tilde}\index{~~@{\tt \protect\Tilde\protect\Tilde}}
           {\em variable-name} {\em variable-default} {\tt]}]~\\
This expression converts to the same tree that the {\em variable-default}
converts to, and adds the root node of this tree to the default set of
the named variable.  Thus \verb|~~| is like \verb|//| except that the node is
added to the default set instead of the value set of the variable.

\item[{\tt [}\ttkey{\#\#\#\#}
      \{ {\em arrow-label} {\em target-value} \}$^\star${\tt]}]~\\
This expression converts to a tree whose root is an unlabeled node
that is the source of arrows labeled with the {\em arrow-labels}.
The target of each arrow is the tree obtained by converting
the {\em target-value} raw expression that follows the arrow's
{\em arrow-label}.

\item[{\tt [}\ttkey{++++}
      {\em expression-graph} {\em expression-graph}$^\star${\tt]}]~\\
This expression converts to the tree that the first {\em expression-graph}
converts to.  The other {\em expression-graphs} are also converted to trees.
All the {\em expression-graphs} may contribute variables, variable values,
and variable defaults, and the values and defaults for each variable
will be merged by the unification at the end of the conversion.

\end{list}
\end{indpar}

After the expression tree is created, its root node becomes the root
of the expression graph resulting from conversion.

Then pure unification is performed as
described in \Secref{PURE-UNIFICATION}, except that the merge set is initialized
to merge pairs that merge the values of each variable and the
defaults of each variable.  It is a conversion error if this
unification fails or if any variable has an empty set of values
before unification is attempted.  After unification a variable has a default
if and only if its set of default values before unification was non-empty.

There are two kinds of graph creation conversions: explicit and implicit.
In \ikey{implicit conversions}{implicit conversion!to expression graph}
what appear to be nodes labeled by a word L beginning with
a capital letter or with `\ttkey{\#}' followed by a capital
letter are treated as the equivalent of `\verb|[// |L\verb|]|'.
Thus L is a variable name and its name used like a node label
in the raw expression
represents a null node that is a value of the variable named L.
For \ikey{explicit conversions}{explicit conversion!to expression graph}
no such special provision is made for words that are used like node labels.

The exact operation names used in graph creation conversions may be different
from those given above.  The names to be used are given as arguments to the
conversion operations, with the names used above as defaults.  These
default names and the descriptive names of the operations are:

\begin{center}
\begin{tabular}{l@{~~~~~}l}
\verb|//|	& \key{graph variable value operation} \\
\verb|~~|	& \key{graph variable default operation} \\
\verb|####|	& \key{graph source construction operation} \\
\verb|++++|	& \key{graph concatenation operation} \\
\end{tabular}
\end{center}

\ikey{Pseudo-variable}{expression graph pseudo-variable}%
\index{pseudo-variable!expression graph}
names (\pagref{PSEUDO-VARIABLES})
of the form `\verb|{|{\em word-or-separator}\verb|}|' can be used
with the `\verb|//|' operation
to merge nodes that are not the value or default of any real variable.
This use simply
creates variables with names that are not single words or separators
which are discarded at the end of conversion.
The pseudo-variable `\verb|{ROOT}|' is always given
the root node of the expression tree as one of its values, but may
be given other values by this method.
Pseudo-variables cannot be given defaults.

Some examples of expression trees represented by raw expressions and the
expression graphs they convert to are as follows:

\begin{indpar}

\begin{tabular}{p{2.5in}@{~~~~~}l}
\begin{tabular}{l}
\verb|[z [// X foo] X]| \\
implicit conversion
\end{tabular}
&
\begin{Graph}
    \Gvariable{\CurlyBra ROOT\CurlyKet} \\
    \begin{Gsingle}{-ARITY-}
	\Glabel{2} \\
    \end{Gsingle} \\
    \begin{Gsingle}{0}
	\Glabel{z} \\
    \end{Gsingle} \\
    \begin{Gchild}{1}
	\Gvariable{X} \\
	\Glabel{foo}
    \end{Gchild} \\
    \begin{Gsingle}{2}
	\Gvariable{X} \\
    \end{Gsingle} \\
\end{Graph}
\end{tabular}

\medskip

\begin{tabular}{p{2.5in}@{~~~~~}l}
\begin{tabular}{l}
\verb|[z [// X foo] X]| \\
explicit conversion
\end{tabular}
&
\begin{Graph}
    \Gvariable{\CurlyBra ROOT\CurlyKet} \\
    \begin{Gsingle}{-ARITY-}
	\Glabel{2} \\
    \end{Gsingle} \\
    \begin{Gsingle}{0}
	\Glabel{z} \\
    \end{Gsingle} \\
    \begin{Gchild}{1}
	\Gvariable{X} \\
	\Glabel{foo}
    \end{Gchild} \\
    \begin{Gsingle}{2}
	\Glabel{X} \\
    \end{Gsingle} \\
\end{Graph}
\end{tabular}

\medskip

\begin{tabular}{p{2.5in}@{~~~~~}l}
\begin{tabular}{@{}l@{}}
\verb|[++++| \\
\verb|  [z X X]| \\
\verb|  [~~ X foo]]| \\[1ex]
implicit conversion
\end{tabular}
&
\begin{Graph}
    \Gvariable{\CurlyBra ROOT\CurlyKet} \\
    \begin{Gsingle}{-ARITY-}
	\Glabel{2} \\
    \end{Gsingle} \\
    \begin{Gsingle}{0}
	\Glabel{z} \\
    \end{Gsingle} \\
    \begin{Gsingle}{1}
	\Gvariable{X} \\
    \end{Gsingle} \\
    \begin{Gsingle}{2}
	\Gvariable{X} \\
    \end{Gsingle} \\
\end{Graph}
~~~
\begin{Graph}
    \Gdefault{X} \\
    \Glabel{foo} \\
\end{Graph}
\end{tabular}

\medskip

\begin{tabular}{p{2.5in}@{~~~~~}l}
\begin{tabular}{@{}l@{}}
\verb|[++++| \\
\verb|  [#### fee X fie X]| \\
\verb|  [// X 51]| \\
\verb|  [~~ X [#### 1 foe 2 fum]]| \\
\verb|  [~~ Y| \\
\verb|    [// Y| \\
\verb|      [#### 1 [// {1}]| \\
\verb|            2 [####| \\
\verb|                *| \\
\verb|                [// {1}]]| \\
\verb|            ]]]]| \\[1ex]
implicit conversion
\end{tabular}
&
\begin{tabular}{@{}l@{}}
\begin{Graph}
    \Gvariable{\CurlyBra ROOT\CurlyKet} \\
    \begin{Gchild}{fee}
	\Gvariable{X} \\
	\Glabel{51} \\
    \end{Gchild} \\
    \begin{Gsingle}{fie}
	\Gvariable{X} \\
    \end{Gsingle} \\
\end{Graph}
\hspace{0.1in}
\begin{Graph}
    \Gdefault{X} \\
    \begin{Gsingle}{1}
	\Glabel{foe} \\
    \end{Gsingle} \\
    \begin{Gsingle}{2}
	\Glabel{fum} \\
    \end{Gsingle} \\
\end{Graph}
\\
~\\
\begin{Graph}
    \Gvariable{Y} \\
    \Gdefault{Y} \\
    \begin{Gsingle}{1}
	\Gvariable{\protect\CurlyBra 1\protect\CurlyKet} \\
    \end{Gsingle} \\
    \begin{Gchild}{2}
	\begin{Gsingle}{*}
	    \Gvariable{\protect\CurlyBra 1\protect\CurlyKet} \\
	\end{Gsingle} \\
    \end{Gchild} \\
\end{Graph}
\end{tabular}

\end{tabular}
\end{indpar}

\subsection{Expression Graph Implementation}
\label{EXPRESSION-GRAPH-IMPLEMENTATION}

In this section we give ideas that may or may not be used to
implement PCASL expression graphs.

PCASL may have a single memory of nodes and arrows that holds all expression
graphs.  We will call this the expression graph memory.

There may be at most one node in expression graph memory that has a given
label L.  That is, all nodes labeled L, for a particular L, are the same
node.

Giving a null node a label L is then implemented by forwarding the null node
to the node with label L.

A frequent operation is copying expression graphs as part of expression
evaluation (\secref{EVALUATION-ALGORITHM}).  To make this more efficient,
an expression graph can be separated into two parts: an immutable part,
and a mutable part.  The mutable part is just a vector of node references,
one per node.  The immutable part is just a set of immutable nodes, with
associated arrows, except that arrow targets which are not labeled nodes
are indices of node cells in the mutable vector.  Thus there are now two
kinds of nodes in memory: normal or mutable nodes, and immutable nodes,
and the latter can only be pointed at by a node reference in a mutable
vector, and in turn can only point at node references in that vector or
at labeled nodes.

Node references are like nodes that are always forwarded.  They
can hold a pointer to an immutable node, or can hold a normal node forwarding
pointer.  They can be pointed at by other nodes.  However, a pointer to
a node reference must include a pointer to its containing mutable
vector, as that vector is needed to interpret any immutable node that the
node reference points at.

The node reference vector for an expression graph also has a node
reference for each expression graph variable value and
for each existing variable default.

During unification arrows are sometimes added to a node.  To make this
more efficient, a node reference can also be permitted to hold a pointer
to a structure that stores
a pointer to an immutable node plus a list of extra arrows added to that
node.  


\section{Expression Evaluation}
\label{EXPRESSION-EVALUATION}

An expression is evaluated by first searching for an expression definition
that matches the expression.  Expression definitions also have guards, which
are expressions that must evaluate to true in order for the expression
definition to apply.  An expression definition may have an expression block
that executes in order to produce a value for the expression if the definition
applies.  If an applicable
expression definition has no expression block, the expression
evaluates to `\ttkey{true}'.  If no expression definition applies to an
expression, the expression evaluates to `\ttkey{false}'.

Expression definitions are searched for in a context, which is a list
of expression definitions and pointers to other contexts.
Each point in the program has a lexical context, which is used by default
for expression evaluation.  There is also a global context to which
definitions may be added or from which they may be deleted.
The block of an expression definition usually evaluates expressions in the
lexical context of its expression definition,
but may evaluate expressions in the context of the expression being
evaluated by the expression definition.

When an expression definition applies to an expression, variables in a copy
of the expression definition are bound to subexpressions of the expression.
Then some of these subexpressions are evaluated and replaced by their
values before the guards and block of the expression definition are
evaluated.  These argument evaluations and also all guard evaluations are
required to have no visible side effects.

\subsection{Expression Definitions}
\label{EXPRESSION-DEFINITIONS}

\ikey{Expression definitions}{expression definition} have the syntax:

\begin{indpar}
\key{expression-definition} ::=
	{\em pattern} \verb|<--| {\em guard-list-option} {\em block-option}
	\\[1ex]
\key{pattern} ::= {\em expression} \\[1ex]
\key{guard-list-option} ::= {\em empty} $|$ {\em guard-list} \\[1ex]
\key{guard-list} ::= {\em guard} \{ \verb|,| {\em guard} \}$^\star$ \\[1ex]
\key{guard} ::= {\em expression} \\[1ex]
\key{block-option} ::= {\em empty} $|$ {\em block} \\[1ex]
\key{block} ::= \verb|{| {\em statement}$^\star$ \verb|}|
\end{indpar}

{\em Blocks} and {\em statements} are further defined in
\secref{BLOCK-EVALUATION}.

In use, the expression definition is copied, and the copy's {\em pattern}
is call unified (\secref{CALL-UNIFICATION}) with the expression being
evaluated.  If unification fails, the definition does not apply.  If
unification succeeds, subexpressions bound to some of the variables in
the {\em pattern} are evaluated and replaced by their values, and then
the copied {\em guards} are evaluated.  If any of the {\em guards} evaluate
to `\verb|false|', the definition does not apply.  If all evaluate to
`\verb|true|', the definition applies.  Then if the definition has
a {\em block}, that is evaluated to produce a value for the expression.
If not, the value of the expression is `\verb|true|'.  If no definition
that applies can be found, the value of the expression is `\verb|false|'.

An expression definition is an expression graph which may have variables
as well as graph nodes.   When an expression definition is copied,
the copy has its own variables that are distinct from the variables in
the original expression definition or in any other expression definition copy.

TBD: how are expressions that have expression definitions defined syntactically
in this document.

\subsection{The Evaluation Algorithm}
\label{EVALUATION-ALGORITHM}

The \key{evaluation algorithm} inputs an expression to be evaluated
and a context.  The context (\secref{CONTEXTS}) provides a list of expression
definitions that may be used to evaluate the expression.

An expression to be evaluated may contain variables that are assigned
values during unification.  The result of evaluation is both a value for
the expression and an assignment of values for these variables.

More than one definition may apply to evaluate an expression.
More than one definition may apply to evaluate a guard of a definition,
and different guard definitions may lead to different values of variables
in the expression being evaluated.  In applying a definition, more than
one argument order assignment (\pagref{ARGUMENT-ORDER-ASSIGNMENT})
may allow the definition to be applied.  So expression
evaluation is a search process with more than one possible choice that
may lead to success.

Note that the guards of an expression are always evaluated from left to
right, so order of guard evaluation is not a choice.

Evaluation may be done in any of the following modes:

\begin{list}{}{}

\item[\ttkey{first-value}]~\\
The first definitions in contexts and first
argument order assignments tried that lead to success are the only ones
used.  Definitions are tried in the order they are given in the contexts
used in evaluation.
The order in which argument order assignments are tried is implementation
dependent.

\item[\ttkey{all-values}]~\\
All possible choices are tried and the
successful results are collected in a set of results.  Each result
in this set consists of a value for the expression being evaluated
and values for each variable in that expression.

\item[\ttkey{consistent-values}]~\\
All possible choices are tried and the
successful results are collected in a set of values.  Then these results
are tested to see if they are pairwise equal.
All the values of the expression
being evaluated must be equal, and all the values of each variable
contained in the expression must be equal.  If yes, one of the results
is used to return the expression and variable values.
If no, an error value describing
the evaluation is returned as the expression value, and as the value of any
variable in the expression.

\item[\ttkey{consistent-reordering}]~\\
Definitions are 
tried in the order they are given in the contexts used in evaluation,
and the first definitions that apply are used.  However, for each definition
all possible argument order assignments are tried.  The results are collected
in a set of results that is tested for pairwise equality.
All the values of the expression
being evaluated must be equal, and all the values of each variable
contained in the expression must be equal.  If yes, one of the results
is used to return the expression and variable values.
If no, an error value describing
the evaluation is returned as the expression value, and as the value of any
variable in the expression.

\end{list}

If the expression evaluation search process yields no results at all,
the expression is given the value `\verb|false|', and variables in the
expression are not changed.

The matching of an expression to be evaluated to a particular
expression definition with a particular argument order
assignment is as follows.

\begin{enumerate}

\item
Make a copy of the expression definition.  Variables in this copy are
distinct from other variables of the same name in memory.

\item

Call unify (\secref{CALL-UNIFICATION})
the pattern (\secref{EXPRESSION-DEFINITIONS})
from the definition copy with the expression to be evaluated.
Both the pattern and expression are expression graphs.
If the unification fails, the definition does not apply.  If the
unification succeeds, it modifies nodes in the expression 
being evaluated and provides variable values for the variables in
the pattern.

\item
For any pattern variable whose name does \underline{not} begin
with `\ttkey{\#}', evaluate the value of that variable to obtain
a new variable value.  Then simultaneously replace all the evaluated
variable values by their new values.  Each replacement is done by
forwarding a variable value graph node to the new value graph node.

These variable value evaluations here are not permitted to have
visible side effects (\secref{VISIBLE-SIDE-EFFECTS}).

\item
Evaluate each guard of the definition copy in left to right order.
If any fail to evaluate to
`\verb|true|', the definition does not apply.  These guard evaluations are not
permitted to have visible side effects (\secref{VISIBLE-SIDE-EFFECTS}).

\item
If the definition applies, but has no block, the value of the expression
is `\verb|true|'.  If the definition applies and has a block, the block
is evaluated (\secref{BLOCK-EVALUATION}) to determine the value of the
expression.

\item
If the definition does not apply, any expression graph changes made while
trying to apply the definition are undone.  Note that invisible side effects
are not undone.

\end{enumerate}

\subsection{Contexts}
\label{CONTEXTS}

Design:

There are immutable contexts that are just first element, rest of context
pairs.  These can be created, or the rest-of-context extracted.

There are mutable contexts that are doubly linked lists which can be iterated
over and have elements deleted.

Context elements can be expression definitions or pointers at contexts (mutable
or immutable).


\section{Block Evaluation}
\label{BLOCK-EVALUATION}

\section{Side Effects}
\label{SIDE-EFFECTS}

Design:

Operations that have side effects can be controlled by side effects modes.

They can be made illegal by setting the illegal-side-effects mode.

They can be encapsulated in a dynamic block that makes included side effects
legal even when side effects outside the block are not.  The included side
effects are called invisible side effects, while those outside the block
are called visible.

Side effects can be delayed.  They are pushed to a FIFO, which can be executed
all at once.

\section{Debugging}

Design:

Debugging is based on the notion that almost all
PCASL programs will run quickly.
Input checkpointing is used to record all inputs to a computation
so the computation can be deterministically rerun.  Detailed
traces can be generated which explain for each value how it was generated.
Values have a sequence number that identifies the point in the execution
where they were generated.  It is therefore possible to ask for a detailed
accounting of how any value was generated, provided the run is short enough
to be repeated once or a few times so the computer can turn the history
tracing on appropriately.

\section{Design Notes}

These are some detailed design rules to be incorporated in the language
definition later.

\begin{list}{}{}

\item
{\bf Quotes}.  \verb|`'| are word, phrase, sentence, paragraph quotes.
\verb|[]| are math expression quotes.   Quotes are in left-right pairs
and can be nested.

\item
{\bf Multiple Quotes}.  Quotes can be multiplied $N$ times: e.g.,
\verb|[[...]]|, \verb|[[[...]]]|, etc.  It is possible to begin a quoted
phrase with a quoted phrase by following the initial $N$-left-quote by
space.  E.g., \verb|`` `Hello' 'tis a good word.''|.

\item
{\bf Algebraic Expressions}.
An expression may or may not be algebraic.  Algebraic expressions have
operators that determine structure.  Atoms in these are sequences of
words, numbers, and parenthesized or bracketed subexpressions.

\item
{\bf N-tuples and Flattening}.
The comma operator forms N-tuples, e.g., \verb|(x,y,z)| is a 3-tuple.

Tuples flatten.  It is not possible for a tuple to be a component of
a tuple: instead the tuples are flattened.  E.g., \verb|(x,(y,z),w)|
is the same value as \verb|(x,y,z,w)|.  There is no such thing as a
1-tuple, but any non-tuple behaves like a 1-tuple.

\item
{\bf Blank Algebraic Operators}.
The blank, or missing operator, denotes \verb|*| when it precedes a word,
as in \verb|5x| being equivalent to \verb|5*x|.  When it is between a
preceding integer and a following ratio, it adds the ratio to the integer,
as in \verb|41 1/3|.

\end{list}



\bibliographystyle{plain}
\bibliography{pcasl}

\printindex

\end{document}

